\documentclass{article}

\usepackage[american]{babel}
\usepackage{csquotes}

\usepackage[
  landscape,
  margin=0.5in,
  footskip=0.25in
]{geometry}
\parindent=0sp
\parskip=0.25cm
\topskip=1sp

\usepackage[
  backend=biber,
  isbn=false
]{biblatex-chicago}
\bibliography{instrumentation-cheat-sheet}

\usepackage{array}
\usepackage{booktabs}
\usepackage{datetime2}
\usepackage{fancyhdr}
\usepackage{multicol}
\usepackage{multirow}

\usepackage[no-config]{fontspec}
\setmainfont{Libertinus Sans}[Numbers=Proportional]
\newfontfamily\symbolFont{Libertinus Serif}

\usepackage{newunicodechar}
\newunicodechar{♯}{{\symbolFont\char`\♯}}
\newunicodechar{♭}{\hspace{-0.15ex}{\symbolFont\char`\♭}\hspace{-0.15ex}}

\usepackage{tikz}
\usetikzlibrary{calc}
\usetikzlibrary{positioning}

\definecolor{primary-ambitus-color}{gray}{0.55}
\definecolor{secondary-ambitus-color}{gray}{0.72}
\definecolor{ambitus-bracket-color}{gray}{0.5}
\definecolor{white-note-guide-color}{gray}{0.97}
\definecolor{black-key-color}{gray}{0.3}
\definecolor{black-note-guide-color}{gray}{0.93}
\definecolor{c-label-color}{gray}{0.7}

\tikzset{instrument label/.style={}}
\tikzset{annotation/.style={font=\itshape, text badly ragged}}

% Change this to 2 for Ableton Live octave numbers.
\def\octaveOffset{1}

% Useful for entering ambitus
\newtoggle{showMIDINoteNumbers}
% \toggletrue{showMIDINoteNumbers}

\usepackage[
  pdftitle={Instrumentation Cheat Sheet},
  pdfauthor={Nathan Whetsell},
  hidelinks
]{hyperref}

\makeatletter
\renewcommand\section{\@startsection {section}{1}{\z@}%
                                     {-3.5ex \@plus -1ex \@minus -.2ex}%
                                     {1sp \@plus .2ex}%
                                     {\normalfont\normalsize}}%
\makeatother

\begin{document}
\pagenumbering{gobble}
\pagestyle{fancyplain}
\fancyhf{}
\renewcommand{\headrulewidth}{0sp}
\renewcommand{\footrulewidth}{0sp}
\rfoot{\fancyplain{}{\tiny\DTMsetstyle{iso}\today}}

\section*{Instrumentation Cheat Sheet}

\def\totalNoteCount{90}
\pgfmathsetmacro{\blackKeyWidth}{\textwidth / \totalNoteCount}

\begin{tikzpicture}[
  every node/.style={
    node font=\tiny,
    inner sep=0,
    outer sep=0
  },
  x=\blackKeyWidth
]
  \input .note-metrics
  \pgfmathsetmacro{\noteImageScaleFactor}{(87 * \blackKeyWidth) / (\maxNoteMidpoint - \minNoteMidpoint)}

  \def\ambitusBarHeight{0.155}
  \def\ambitusBarSeparation{0.45}

  \newdimen\notesTopHeight
  \settoheight{\notesTopHeight}{\includegraphics[scale=\noteImageScaleFactor]{notes-top.pdf}}

  % Number of instruments, except strings and timpani
  \def\separatedInstrumentCount{35}

  \pgfmathsetmacro{\noteGuideTop}{
    (
      \separatedInstrumentCount + (\separatedInstrumentCount - 4) * \ambitusBarSeparation +
      % strings
      4 * (6 + 0.25) +
      % timpani
      5 +
      % instrument family separation
      4 * 2
    ) * \ambitusBarHeight +
    \notesTopHeight * 2.54/72.27
  }

  \def\bottomNotePadding{0.1}
  \pgfmathsetmacro{\noteGuideBottom}{\noteImageScaleFactor * -\minNoteY * 2.54/72.27 + \bottomNotePadding}

  \def\keyboardHeight{1}

  \path[use as bounding box] (0, -\noteGuideBottom) rectangle (\totalNoteCount * \blackKeyWidth, \keyboardHeight + \noteGuideTop);

  % Draw white note guides.
  \NewDocumentCommand\AToDGuidePaths{}{
    % A
    rectangle ++(1, -\noteGuideBottom) ++(-1,  \noteGuideBottom) ++(0,  \keyboardHeight)
    rectangle ++(1,  \noteGuideTop) ++(-1, -\noteGuideTop) ++(2, -\keyboardHeight)
    % B
    rectangle ++(1, -\noteGuideBottom) ++(-1,  \noteGuideBottom) ++(0,  \keyboardHeight)
    rectangle ++(1,  \noteGuideTop) ++(-1, -\noteGuideTop) ++(3, -\keyboardHeight)
    % D
    rectangle ++(1, -\noteGuideBottom) ++(-1,  \noteGuideBottom) ++(0,  \keyboardHeight)
    rectangle ++(1,  \noteGuideTop) ++(-1, -\noteGuideTop) ++(2, -\keyboardHeight)
  }
  \pgfmathsetmacro{\lastOctave}{div(\totalNoteCount, 12) - 1}
  \def\EFSeparation{0.05}
  \filldraw[
    draw=white-note-guide-color,
    fill=white-note-guide-color
  ]
    (0, 0)
    \foreach \octave in { 0, ..., \lastOctave } {
      \AToDGuidePaths
      % E
      rectangle ++(1 - \EFSeparation, -\noteGuideBottom) ++(-1 + \EFSeparation,  \noteGuideBottom) ++(0,  \keyboardHeight)
      rectangle ++(1 - \EFSeparation,  \noteGuideTop) ++(-1 + \EFSeparation, -\noteGuideTop) ++(1 + \EFSeparation, -\keyboardHeight)
      % F
      rectangle ++(1 - \EFSeparation, -\noteGuideBottom) ++(-1 + \EFSeparation,  \noteGuideBottom) ++(0,  \keyboardHeight)
      rectangle ++(1 - \EFSeparation,  \noteGuideTop) ++(-1 + \EFSeparation, -\noteGuideTop) ++(2 - \EFSeparation, -\keyboardHeight)
      % G
      rectangle ++(1, -\noteGuideBottom) ++(-1,  \noteGuideBottom) ++(0,  \keyboardHeight)
      rectangle ++(1,  \noteGuideTop) ++(-1, -\noteGuideTop) ++(2, -\keyboardHeight)
    }
    \AToDGuidePaths;

  % Draw black note guides.
  \NewDocumentCommand\BFlatToCSharpGuidePaths{}{
    % A♯/B♭
    rectangle ++(1, -\noteGuideBottom) ++(-1,  \noteGuideBottom) ++(0,  \keyboardHeight)
    rectangle ++(1,  \noteGuideTop) ++(-1, -\noteGuideTop) ++(3, -\keyboardHeight)
    % C♯/D♭
    rectangle ++(1, -\noteGuideBottom) ++(-1,  \noteGuideBottom) ++(0,  \keyboardHeight)
    rectangle ++(1,  \noteGuideTop) ++(-1, -\noteGuideTop) ++(2, -\keyboardHeight)
  }
  \filldraw[
    draw=black-note-guide-color,
    fill=black-note-guide-color
  ]
    (1, 0)
    \foreach \octave in { 0, ..., \lastOctave } {
      \BFlatToCSharpGuidePaths
      % D♯/E♭
      rectangle ++(1, -\noteGuideBottom) ++(-1,  \noteGuideBottom) ++(0,  \keyboardHeight)
      rectangle ++(1,  \noteGuideTop) ++(-1, -\noteGuideTop) ++(3, -\keyboardHeight)
      % F♯/G♭
      rectangle ++(1, -\noteGuideBottom) ++(-1,  \noteGuideBottom) ++(0,  \keyboardHeight)
      rectangle ++(1,  \noteGuideTop) ++(-1, -\noteGuideTop) ++(2, -\keyboardHeight)
      % G♯/A♭
      rectangle ++(1, -\noteGuideBottom) ++(-1,  \noteGuideBottom) ++(0,  \keyboardHeight)
      rectangle ++(1,  \noteGuideTop) ++(-1, -\noteGuideTop) ++(2, -\keyboardHeight)
    }
    \BFlatToCSharpGuidePaths;

  % Draw the keyboard.
  \pgfmathsetmacro{\blackKeyHeight}{0.625 * \keyboardHeight}
  \pgfmathsetmacro{\whiteToBlackKeyDistance}{\keyboardHeight - \blackKeyHeight}
  \begin{scope}[
    draw=black-key-color,
    fill=black-key-color
  ]
    \NewDocumentCommand\AToCSharpKeyPaths{}{
      ++( 1.5, 0) -- ++(0, \whiteToBlackKeyDistance)
      ++(-0.5, 0) rectangle ++(1, \blackKeyHeight) % A♯/B♭
      ++( 1  , -\keyboardHeight) -- ++(0, \keyboardHeight)
      ++( 1.5, -\keyboardHeight) -- ++(0, \whiteToBlackKeyDistance)
      ++(-0.5, 0) rectangle ++(1, \blackKeyHeight) % C♯/D♭
    }
    \filldraw (0, 0)
    \foreach \octave in { 0, ..., \lastOctave } {
      \AToCSharpKeyPaths
      ++( 1.5, -\keyboardHeight) -- ++(0, \whiteToBlackKeyDistance)
      ++(-0.5, 0) rectangle ++(1, \blackKeyHeight) % D♯/E♭
      ++( 1  , -\keyboardHeight) -- ++(0, \keyboardHeight)
      ++( 1.5, -\keyboardHeight) -- ++(0, \whiteToBlackKeyDistance)
      ++(-0.5, 0) rectangle ++(1, \blackKeyHeight) % F♯/G♭
      ++( 1.5, -\keyboardHeight) -- ++(0, \whiteToBlackKeyDistance)
      ++(-0.5, 0) rectangle ++(1, \blackKeyHeight) % G♯/A♭
      ++( 0  , -\keyboardHeight)
    }
    \AToCSharpKeyPaths;

    \draw (0, 0) rectangle ++(\totalNoteCount, \keyboardHeight);
  \end{scope}

  % Label notes.
  \begin{scope}[
    anchor=north west,
    minimum width=\blackKeyWidth,
    text=c-label-color
  ]
    % Label Cs.
    \foreach \noteNumber in { 24, 36, ..., 108 } {
      \pgfmathsetmacro{\octave}{int(div(\noteNumber, 12) - \octaveOffset)}
      \path
        (\noteNumber - 21, \keyboardHeight)
        node[yshift=-0.75ex] { C\octave }
        ++(0, \noteGuideTop)
        node { C\octave };
    }
    % Label concert pitch (A above middle C).
    \def\noteNumber{69}
    \pgfmathsetmacro{\octave}{int(div(\noteNumber, 12) - \octaveOffset)}
    \path
      (\noteNumber - 21, \keyboardHeight)
      node[yshift=-0.75ex] { A\octave }
      ++(0, \noteGuideTop)
      node { A\octave };
  \end{scope}

  % Add MIDI note numbers.
  \iftoggle{showMIDINoteNumbers}{
    \begin{scope}[
      anchor=mid west,
      minimum width=\blackKeyWidth,
    ]
      \foreach \noteNumber in { 21, ..., 110 } {
        \pgfmathsetmacro{\pitchNumber}{mod(\noteNumber, 12)}
        \pgfmathparse{
          \pitchNumber ==  1 ||
          \pitchNumber ==  3 ||
          \pitchNumber ==  6 ||
          \pitchNumber ==  8 ||
          \pitchNumber == 10
        }
        \ifnum\pgfmathresult>0\relax
          \def\noteNumberColor{white}
        \else
          \def\noteNumberColor{black}
        \fi
        \node[text=\noteNumberColor] at (\noteNumber - 21, 0.5*\keyboardHeight) { \noteNumber };
      }
    \end{scope}
  }{}

  \pgfmathsetmacro{\labelOffset}{-0.1 * \blackKeyWidth}
  \gdef\ambitusBarOffset{0}

  \ExplSyntaxOn

  \seq_const_from_clist:Nn \note_names { C, D♭, D, E♭, E, F, F♯, G, A♭, A, B♭, B }
  \NewDocumentCommand { \noteName } { m }
    {
      \seq_item:Nn \note_names { \int_mod:nn { #1 } { 12 } + 1 }
      \int_eval:n { \int_div_truncate:nn { #1 } { 12 } - \octaveOffset }
    }

  \def\lowNodeName{}
  \def\highNodeName{}

  \NewDocumentCommand\drawAmbitus{s m o m o m o o}{
    \IfBooleanF{#1}{
      \pgfmathsetmacro{\ambitusBarOffset}{\ambitusBarOffset + 1 + \ambitusBarSeparation}
      \xdef\ambitusBarOffset{\ambitusBarOffset}
    }
    \pgfmathsetmacro{\bottomNote}{#4 - 21}
    \def\noteCount{\numexpr #6 - #4 + 1}
    \IfValueT{#7}{
      \tl_set:Nn \l_tmpa_tl { #7 }
      \tl_replace_all:Nnn \l_tmpa_tl { ♭ } { -flat }
      \tl_remove_all:Nn \l_tmpa_tl { . }
      \tl_remove_all:Nn \l_tmpa_tl { ( }
      \tl_remove_all:Nn \l_tmpa_tl { ) }
      \edef\lowNodeName{\l_tmpa_tl-low}
      \edef\highNodeName{\l_tmpa_tl-high}
    }

    \begin{scope}[node~ distance=0]
      \path (\bottomNote, \keyboardHeight + \ambitusBarOffset * \ambitusBarHeight)
        node[anchor=south~ west, text=white] (\lowNodeName) {
          \IfValueTF{#3}{#3}{\noteName{#4}}
        }
        node[base~ left=of~ \lowNodeName, xshift=\labelOffset, instrument~ label] { \IfValueT{#7}{#7} }
        [fill=#2, rounded~ corners=0.5bp] rectangle ++(\noteCount, \ambitusBarHeight)
        node[anchor=north~ east, text=white] (\highNodeName) {
          \ifnum \noteCount>1\relax
            \IfValueTF{#5}{#5}{\noteName{#6}}
          \fi
        };
    \end{scope}
  }

  \ExplSyntaxOff

  \NewDocumentCommand\drawPrimaryAmbitus{s o m o m o}{
    \IfBooleanTF{#1}
      {\drawAmbitus*{primary-ambitus-color}[#2]{#3}[#4]{#5}[#6]}
      {\drawAmbitus{primary-ambitus-color}[#2]{#3}[#4]{#5}[#6]}
  }
  \NewDocumentCommand\drawSecondaryAmbitus{s o m o m o}{
    \IfBooleanTF{#1}
      {\drawAmbitus*{secondary-ambitus-color}[#2]{#3}[#4]{#5}[#6]}
      {\drawAmbitus{secondary-ambitus-color}[#2]{#3}[#4]{#5}[#6]}
  }

  \NewDocumentEnvironment{instrumentFamily}{}{
    \def\defaultInstrumentFamilySeparation{1.5}
    \gdef\instrumentFamilySeparation{\defaultInstrumentFamilySeparation}
  }{
    \pgfmathsetmacro{\ambitusBarOffset}{\ambitusBarOffset + \instrumentFamilySeparation}
    \xdef\ambitusBarOffset{\ambitusBarOffset}

    \gdef\instrumentFamilySeparation{\defaultInstrumentFamilySeparation}
  }

  \NewDocumentEnvironment{bracketedAmbitus}{m o}{
    \def\ambitusBarSeparation{0.25}

    \newtoggle{didRun}

    \NewCommandCopy\drawAmbitusOriginal\drawAmbitus
    \RenewDocumentCommand\drawAmbitus{s m o m o m o}{
      \iftoggle{didRun}{
        \def\lowNodeName{}
        \def\highNodeName{high}
      }{
        \def\lowNodeName{low}
        \def\highNodeName{}
        \toggletrue{didRun}
      }

      \IfBooleanTF{##1}
        {\drawAmbitusOriginal*{##2}[##3]{##4}[##5]{##6}[##7]}
        {\drawAmbitusOriginal{##2}[##3]{##4}[##5]{##6}[##7]}
    }

    \begin{scope}[name prefix = #1-]
  }{
    \end{scope}

    \path[draw=ambitus-bracket-color]
      let \p1=(#1-low), \p2=(#1-high) in
      (\x1 - 0.25 * \blackKeyWidth, \y1 - \ambitusBarHeight cm) --
      (\x1 - 0.75 * \blackKeyWidth, \y1 - \ambitusBarHeight cm) --
      (\x1 - 0.75 * \blackKeyWidth, \y2 + 0.5 * \ambitusBarHeight cm)
      node[midway, anchor=mid east, xshift=\labelOffset] (#1-midpoint) { #1 \IfValueT{#2}{#2} } --
      (\x1 - 0.25 * \blackKeyWidth, \y2 + 0.5 * \ambitusBarHeight cm)
      % let \p3=(#1-midpoint) in
      % (\x1 - 0.75 * \blackKeyWidth, \y3) -- (\x1 - \blackKeyWidth, \y3)
      ;

    \RenewCommandCopy\drawAmbitus\drawAmbitusOriginal

    \pgfmathsetmacro{\ambitusBarOffset}{\ambitusBarOffset + 1}
    \xdef\ambitusBarOffset{\ambitusBarOffset}

    \gdef\instrumentFamilySeparation{0.5}
  }

  \NewDocumentEnvironment{compactAmbitus}{}{\def\ambitusBarSeparation{0}}{}

  \begin{scope}[
    minimum height=\ambitusBarHeight cm,
    minimum width=\blackKeyWidth
  ]
    \begin{instrumentFamily}
      \begin{bracketedAmbitus}{Contrabass}
        \begin{compactAmbitus}
          \drawSecondaryAmbitus{24}{28}
          \drawSecondaryAmbitus*{40}{52}
          \drawPrimaryAmbitus*{28}{40}
          \drawSecondaryAmbitus{45}{57}
          \drawPrimaryAmbitus*{33}{45}
          \drawSecondaryAmbitus{50}{62}
          \drawPrimaryAmbitus*{38}{50}
          \drawSecondaryAmbitus{54}{67}
          \drawPrimaryAmbitus*{43}{55}
        \end{compactAmbitus}
        \drawPrimaryAmbitus{40}{40}[Cb. harmonics]
        \drawPrimaryAmbitus*{45}{45}
        \drawPrimaryAmbitus*{47}{47}
        \drawPrimaryAmbitus*{50}{50}
        \drawSecondaryAmbitus*{84}{89}
        \drawPrimaryAmbitus*{52}{84}
      \end{bracketedAmbitus}

      \begin{bracketedAmbitus}{Cello}
        \begin{compactAmbitus}
          \drawSecondaryAmbitus{36}{60}
          \drawPrimaryAmbitus*{36}{53}
          \drawSecondaryAmbitus{60}{67}
          \drawPrimaryAmbitus*{43}{60}
          \drawSecondaryAmbitus{67}{74}
          \drawPrimaryAmbitus*{50}{67}
          \drawSecondaryAmbitus{81}{82}
          \drawPrimaryAmbitus*{57}{81}
        \end{compactAmbitus}
        \drawPrimaryAmbitus{48}{48}[Vcl. harmonics]
        \drawPrimaryAmbitus*{55}{55}
        \drawSecondaryAmbitus*{89}{95}
        \drawPrimaryAmbitus*{60}{89}

        \path
          let \p1=(Vcl harmonics-high) in
          (96.5 - 21, \y1 + 0.5*\ambitusBarHeight cm)
          node[annotation, anchor=mid west, text width=14.5*\blackKeyWidth] {
            On all stringed instruments, higher harmonics are possible but not always available in sample sets.
          };
      \end{bracketedAmbitus}

      \begin{bracketedAmbitus}{Viola}
        \begin{compactAmbitus}
          \drawPrimaryAmbitus{48}{72}
          \drawPrimaryAmbitus{55}{79}
          \drawPrimaryAmbitus{62}{86}
          \drawPrimaryAmbitus{69}{93}
        \end{compactAmbitus}
        \drawPrimaryAmbitus{60}{60}[Vla. harmonics]
        \drawPrimaryAmbitus*{67}{67}
        \drawSecondaryAmbitus*{100}{101}
        \drawPrimaryAmbitus*{72}{100}
      \end{bracketedAmbitus}

      \begin{bracketedAmbitus}{Violin}
        \begin{compactAmbitus}
          \drawPrimaryAmbitus{55}{79}
          \drawPrimaryAmbitus{62}{86}
          \drawPrimaryAmbitus{69}{93}
          \drawPrimaryAmbitus{76}{100}
        \end{compactAmbitus}
        \drawPrimaryAmbitus{67}{67}[Vln. harmonics]
        \drawPrimaryAmbitus*{74}{74}
        \drawSecondaryAmbitus*{108}{110}
        \drawPrimaryAmbitus*{79}{108}
      \end{bracketedAmbitus}
    \end{instrumentFamily}

    \begin{instrumentFamily}
      \drawPrimaryAmbitus{21}{108}[Piano]
      \drawPrimaryAmbitus[C♭0]{23}[G♯7]{104}[Harp]
      \drawSecondaryAmbitus{36}{48}[Marimba]
      \drawPrimaryAmbitus*{48}{96}
      \drawSecondaryAmbitus{58}{65}[Xylophone]
      \drawPrimaryAmbitus*{65}{108}
      \drawSecondaryAmbitus{48}{53}[Vibraphone]
      \drawPrimaryAmbitus*{53}{89}
      \drawPrimaryAmbitus{60}{108}[Celesta]
      \drawSecondaryAmbitus{52}{79}[Tubular Bells]
      \drawPrimaryAmbitus*{60}{77}
      \drawSecondaryAmbitus{72}{79}[Glockenspiel]
      \drawPrimaryAmbitus*{79}{108}
      \drawPrimaryAmbitus{84}{108}[Crotales]
      \begin{bracketedAmbitus}{Timpani}
        \begin{compactAmbitus}
          \drawSecondaryAmbitus{36}{45}
          \drawPrimaryAmbitus{41}{48}
          \drawPrimaryAmbitus{46}{53}
          \drawPrimaryAmbitus{50}{57}
          \drawSecondaryAmbitus{53}{61}
        \end{compactAmbitus}
      \end{bracketedAmbitus}
    \end{instrumentFamily}

    \begin{instrumentFamily}
      \drawPrimaryAmbitus{22}{63}[B♭ Tuba]
      \drawPrimaryAmbitus{26}{64}[C Tuba]
      \drawPrimaryAmbitus{35}{77}[F Wagner Tuba]
      \drawPrimaryAmbitus{40}{77}[B♭ Wagner Tuba]
      \drawPrimaryAmbitus{34}{77}[Bass Trombone]
      \drawPrimaryAmbitus{40}{77}[Trombone]
      \drawSecondaryAmbitus{51}{52}[B♭ Trumpet]
      \drawPrimaryAmbitus*{52}{82}
      \drawSecondaryAmbitus{53}{54}[C Trumpet]
      \drawPrimaryAmbitus*{54}{84}
      \drawSecondaryAmbitus{62}{64}[B♭ Piccolo Trumpet]
      \drawPrimaryAmbitus*{64}{91}
      \drawPrimaryAmbitus{31}{77}[French Horn]

      \path
        let \p1=(C Trumpet-low) in
        (0, \y1)
        node[annotation, anchor=north west, text width=7.5*\blackKeyWidth] {
          Lower pedal tones are possible on brass instruments.
        };
    \end{instrumentFamily}

    \begin{instrumentFamily}
      \drawSecondaryAmbitus{36}{37}[Baritone Saxophone]
      \drawPrimaryAmbitus*{37}{75}
      \drawPrimaryAmbitus{44}{82}[Tenor Saxophone]
      \drawPrimaryAmbitus{49}{87}[Alto Saxophone]
      \drawPrimaryAmbitus{56}{94}[Soprano Saxophone]

      \path
        let \p1=(Soprano Saxophone-high) in
        (\x1 + \blackKeyWidth, \y1 + \ambitusBarHeight cm)
        node[annotation, anchor=mid west, text width=14.5*\blackKeyWidth] {
          Higher notes are possible on wind instruments, but increasingly difficult to sound and not always available in sample sets.
        };
    \end{instrumentFamily}

    \begin{instrumentFamily}
      \tikzset{instrument label/.style={anchor=north west, xshift=-\labelOffset, yshift=-0.2*\ambitusBarHeight cm}}
      \drawSecondaryAmbitus{21}{22}[Contrabassoon]
      \tikzset{instrument label/.style={}}
      \drawPrimaryAmbitus*{22}{60}
      \drawSecondaryAmbitus{34}{77}[Bassoon]
      \drawPrimaryAmbitus*{34}{75}
      \tikzset{instrument label/.style={anchor=north west, xshift=-\labelOffset, yshift=-0.2*\ambitusBarHeight cm}}
      \drawPrimaryAmbitus{22}{67}[Contrabass Clarinet]
      \tikzset{instrument label/.style={}}
      \drawSecondaryAmbitus{34}{83}[Bass Clarinet]
      \drawPrimaryAmbitus*{34}{79}
      \drawSecondaryAmbitus{42}{84}[E♭ Alto Clarinet]
      \drawPrimaryAmbitus*{43}{79}
      \drawPrimaryAmbitus{50}{91}[B♭ Clarinet]
      \drawPrimaryAmbitus{55}{95}[E♭ Clarinet]
      \drawPrimaryAmbitus{52}{83}[English Horn]
      \drawSecondaryAmbitus{58}{96}
      \drawPrimaryAmbitus*{58}{91}[Oboe]
      \drawPrimaryAmbitus{55}{88}[Alto Flute]
      \drawSecondaryAmbitus{59}{60}[Flute]
      \drawPrimaryAmbitus*{60}{98}
      \drawPrimaryAmbitus{74}{108}[Piccolo]
    \end{instrumentFamily}
  \end{scope}

  \path
    let \p1=(Piccolo-high) in
    (0.5 * \blackKeyWidth - \minNoteMidpoint * \noteImageScaleFactor, -\bottomNotePadding)
    node[anchor=north west] {
      \includegraphics[scale=\noteImageScaleFactor]{notes-bottom.pdf}
    }
    ++(0, \y1 - \ambitusBarHeight cm + \bottomNotePadding cm)
    node[anchor=south west] {
      \includegraphics[scale=\noteImageScaleFactor]{notes-top.pdf}
    };
\end{tikzpicture}

\clearpage

\begin{multicols*}{4}
  \scriptsize
  \addfontfeature{Numbers=OldStyle}

  \section*{Instrument Transpositions}

  \begin{tabular}{@{}lll@{}} \toprule
    Instrument          & To Concert Pitch \\ \midrule
    Alto Flute          & 4th down \\
    English Horn        & 5th down \\
    E♭ Clarinet         & minor~3rd up \\
    B♭ Clarinet         & 2nd down \\
    E♭ Alto Clarinet    & minor 6th down \\
    Bass Clarinet       & 2nd + octave down \\
    Contrabass Clarinet & 2nd + 2 octaves down \\
    Contrabassoon       & octave down \\ \midrule
    Soprano Saxophone   & 2nd down \\
    Alto Saxophone      & minor 6th down \\
    Tenor Saxophone     & 2nd + octave down \\
    Baritone Saxophone  & minor 6th + octave down \\ \midrule
    French Horn in F    & \begin{tabular}{@{}l@{}}
                            5th down, treble \& new bass clef \\
                            4th up, old bass clef
                          \end{tabular} \\
    F Wagner Tuba       & 5th down \\
    B♭ Wagner Tuba      & 2nd down \\
    B♭ Piccolo Trumpet  & minor 7th up \\
    B♭ Trumpet          & 2nd down \\ \midrule
    Crotales            & 2 octaves up \\
    Glockenspiel        & 2 octaves up \\
    Celesta             & octave up \\
    Xylophone           & octave up \\ \midrule
    Contrabass          & octave down \\ \bottomrule
  \end{tabular}

  \section*{Note Names}

  Note names use “Yamaha” octaves (the default in Apple Logic, concert pitch is {\addfontfeature{Numbers=Lining}A4}).
  Subtract 1 for “Roland” octaves (used in Ableton Live, concert pitch is {\addfontfeature{Numbers=Lining}A3}).

  \section*{Page Numbers of References}

  For more information about instruments and orchestration:

  \newcolumntype{R}{>{\addfontfeature{Numbers={Lining,Monospaced}}}r}

  \newlength\instrumentWidth
  \settowidth{\instrumentWidth}{Contrabass Clarinet}

  \addtolength{\tabcolsep}{-0.4em}

  % A longtable environment would be preferred, but longtable can’t be used in a
  % multicols environment.
  \begin{tabular}[t]{@{}p{\instrumentWidth}RRR@{}} \toprule
                        & \multicolumn{3}{c}{Page No.} \\ \cmidrule(){2-4}
    Instrument          & \citeauthor{adler}    & \citeauthor{blatter}    & \citeauthor{forsyth} \\ \midrule
    Piccolo             & \notecite[198]{adler} & \multirow{3}{*}{\notecite[90]{blatter}}  & \notecite[198]{forsyth} \\
    Flute               & \notecite[189]{adler} &                         & \notecite[182]{forsyth} \\
    Alto Flute          & \notecite[201]{adler} &                         & \notecite[196]{forsyth} \\ \specialrule{\cmidrulewidth}{0.5\aboverulesep}{0.5\belowrulesep}
    Oboe                & \notecite[204]{adler} & \multirow{2}{*}{\notecite[98]{blatter}}  & \notecite[204]{forsyth} \\
    English Horn        & \notecite[209]{adler} &                         & \notecite[220]{forsyth} \\ \specialrule{\cmidrulewidth}{0.5\aboverulesep}{0.5\belowrulesep}
    E♭ Clarinet         & \notecite[224]{adler} & \multirow{5}{*}{\notecite[105]{blatter}} & \notecite[278]{forsyth} \\
    B♭ Clarinet         & \notecite[217]{adler} &                         & \notecite[251]{forsyth} \\
    E♭ Alto Clarinet    & \notecite[228]{adler} &                         & \notecite[282]{forsyth} \\
    Bass Clarinet       & \notecite[225]{adler} &                         & \notecite[272]{forsyth} \\
    Contrabass Clarinet & \notecite[230]{adler} &                         & \notecite[286]{forsyth} \\ \specialrule{\cmidrulewidth}{0.5\aboverulesep}{0.5\belowrulesep}
    Bassoon             & \notecite[235]{adler} & \multirow{2}{*}{\notecite[116]{blatter}} & \notecite[229]{forsyth} \\
    Contrabassoon       & \notecite[240]{adler} &                         & \notecite[246]{forsyth} \\ \midrule
    Saxophones          & \notecite[231]{adler} & \notecite[126]{blatter} & \notecite[166]{forsyth} \\ \midrule
    French Horn         & \notecite[337]{adler} & \notecite[148]{blatter} & \notecite[109]{forsyth} \\
    Trumpets            & \notecite[357]{adler} & \notecite[159]{blatter} & \notecite[89]{forsyth} \\
    Trombones           & \notecite[368]{adler} & \notecite[169]{blatter} & \notecite[133]{forsyth} \\
    Wagner Tubas        & —                     & \notecite[148]{blatter} & \notecite[153]{forsyth} \\ \bottomrule
  \end{tabular}

  % \vfill\null
  \columnbreak

  \emph{(Page numbers of references, cont.)}\\
  \begin{tabular}[t]{@{}p{\instrumentWidth}RRR@{}} \toprule
                        & \multicolumn{3}{c}{Page No.} \\ \cmidrule(){2-4}
    Instrument          & \citeauthor{adler}    & \citeauthor{blatter}    & \citeauthor{forsyth} \\ \midrule
    Tubas               & \notecite[377]{adler} & \notecite[178]{blatter} & \notecite[151]{forsyth} \\ \midrule
    Timpani             & \notecite[485]{adler} & \notecite[209]{blatter} & \notecite[41]{forsyth} \\ \midrule
    Crotales            & \notecite[481]{adler} & \notecite[206]{blatter} & — \\
    Glockenspiel        & \notecite[479]{adler} & \notecite[205]{blatter} & \notecite[60]{forsyth} \\
    Tubular Bells       & \notecite[480]{adler} & \notecite[205]{blatter} & \notecite[53]{forsyth} \\
    Celesta             & \notecite[528]{adler} & \notecite[206]{blatter} & \notecite[64]{forsyth} \\
    Vibraphone          & \notecite[477]{adler} & \notecite[205]{blatter} & — \\
    Xylophone           & \notecite[475]{adler} & \notecite[204]{blatter} & \notecite[66]{forsyth} \\
    Marimba             & \notecite[476]{adler} & \notecite[204]{blatter} & — \\ \midrule
    Harp                & \notecite[95]{adler}  & \notecite[252]{blatter} & \notecite[461]{forsyth} \\
    Piano               & \notecite[521]{adler} & \notecite[242]{blatter} & — \\ \midrule
    Violin              & \notecite[57]{adler}  & \notecite[49]{blatter}, \notecite[441]{blatter} & \notecite[303]{forsyth} \\
    Viola               & \notecite[71]{adler}  & \notecite[56]{blatter}, \notecite[442]{blatter} & \notecite[381]{forsyth} \\
    Cello               & \notecite[81]{adler}  & \notecite[60]{blatter}, \notecite[443]{blatter} & \notecite[409]{forsyth} \\
    Contrabass          & \notecite[89]{adler}  & \notecite[67]{blatter}, \notecite[444]{blatter} & \notecite[436]{forsyth} \\ \bottomrule
  \end{tabular}

  \nocite{vienna-academy}

  \renewcommand*\bibfont{\scriptsize}
  \printbibliography
\end{multicols*}

\end{document}
