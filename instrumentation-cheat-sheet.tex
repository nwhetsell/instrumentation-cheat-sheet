\documentclass{article}

\usepackage[american]{babel}
\hyphenation{
Fab-Fil-ter
Ya-ma-ha
}
\usepackage{csquotes}

\usepackage[
  landscape,
  margin=0.5in,
  vmargin=0.75in,
  footskip=0.25in
]{geometry}
\parindent=0sp
\parskip=0.125cm
\topskip=1sp

\usepackage[
  backend=biber,
  isbn=false
]{biblatex-chicago}
\bibliography{instrumentation-cheat-sheet}

\usepackage{array}
\usepackage{booktabs}
\usepackage{datetime2}
\usepackage{fancyhdr}
\usepackage{multicol}
\usepackage{multirow}

\usepackage[no-config]{fontspec}
\setmainfont{Libertinus Sans}[Numbers=Proportional]
\setmonofont{TeX Gyre Cursor}
\newfontfamily\symbolFont{Libertinus Serif}

\usepackage{newunicodechar}
\newunicodechar{♯}{{\symbolFont\char`\♯}}
\newunicodechar{♭}{\hspace{-0.15ex}{\symbolFont\char`\♭}\hspace{-0.15ex}}

\usepackage{tikz}
\usetikzlibrary{backgrounds}
\usetikzlibrary{calc}
\usetikzlibrary{positioning}

\definecolor{primary-ambitus-color}  {gray}{0.55}
\definecolor{secondary-ambitus-color}{gray}{0.72}
\definecolor{ambitus-bracket-color}  {gray}{0.5}
\definecolor{white-note-guide-color} {gray}{0.975}
\definecolor{black-key-color}        {gray}{0.3}
\definecolor{black-note-guide-color} {gray}{0.93}
\definecolor{note-label-color}       {gray}{0.7}
\definecolor{frequency-label-color}  {gray}{0.7}
\definecolor{white-note-number-color}{gray}{0.8}
\definecolor{black-note-number-color}{gray}{0.525}

\tikzset{annotation/.style={font=\itshape, text badly ragged}}
\tikzset{instrument label/.style={overlay}}
\tikzset{note name/.style={anchor=north west, text=white}}

\input .octave-offset.tex

\usepackage[
  pdftitle={Instrumentation Cheat Sheet},
  pdfauthor={Nathan Whetsell},
  hidelinks
]{hyperref}

\makeatletter
\renewcommand\section{\@startsection {section}{1}{\z@}%
                                     {-2ex \@plus -1ex \@minus -.2ex}%
                                     {1sp \@plus .2ex}%
                                     {\normalfont\normalsize}}%
\makeatother

\AddToHook{begindocument/end}{
  % Don’t put extra space after a full stop.
  \frenchspacing

  % Put the commit hash and current date in the footer.
  \pagestyle{fancyplain}
  \fancyhf{}
  \renewcommand{\headrulewidth}{0pt}
  \renewcommand{\footrulewidth}{0pt}
  \lfoot{\fancyplain{}{\tiny commit \ttfamily \input|"git rev-parse --short HEAD"}}
  \rfoot{\fancyplain{}{\tiny\DTMsetstyle{iso}\today}}
}

\ExplSyntaxOn

\tl_const:Nn \l_total_note_count_tl { 90 }

% TikZ does not treat coordinates that lack units (the usual case) the same as
% coordinates that have units, so we need to be careful about which is which.
% Use token lists for numbers without units and dimensions for numbers with
% units. (This isn’t a perfect solution. Token lists can contain anything, so a
% token list can become a number with a unit simply by appending “pt”, for
% example.)
\tl_const:Nn  \l_ambitus_bar_height_tl { 0.155 }
\tl_const:Nn  \l_ambitus_bar_separation_tl { 0.45 } % This is a fraction of the ambitus bar height.
\tl_new:N     \g_ambitus_y_tl
% From https://tikz.dev/tikz-actions#sec-15.3.1, TikZ uses 0.4pt for the default
% width of a line. Subtract this from \textwidth so the cheat sheet doesn’t
% extend into the right margin.
\dim_const:Nn \l_black_key_width_dim { (\textwidth - 0.4pt) / \l_total_note_count_tl }

\tikzset{midi~ note~ number/.style={anchor=mid~ west, minimum~ height=\l_ambitus_bar_height_tl cm, minimum~ width=\l_black_key_width_dim}}
\tikzset{note~ name/.append~ style={minimum~ width=\l_black_key_width_dim}}

\pgfkeys{/pgf/number~ format/.cd,
  assume~ math~ mode=true, % Don’t switch to math mode (which would use the wrong font).
  fixed,                   % Round numbers to a fixed number of digits after the decimal point.
  precision=1              % Round to the nearest tenth.
}

% The highest note is the D five half-steps above the A at 3520 Hz, so in most
% DAWs using 12-EDO tuning, the fundamental frequency of the D will be
% 2^(5/12) * 3520 Hz. Use a LaTeX3 floating point expression because the PGF
% math engine is a bit too imprecise to compute this.
\tl_const:Nx \l_max_frequency_tl { \fp_to_decimal:n { 2^(5/12) * 3520 } }
\tl_const:Nn \l_min_frequency_tl { 27.5 }
\bool_new:N \g_did_print_frequency_unit_bool

\NewDocumentEnvironment { instrumentationPicture } { }
  {
    \tl_gset:Nn \g_ambitus_y_tl 0

    \begin{tikzpicture}[
      every~ node/.style={
        node~ font=\tiny,
        inner~ sep=0,
        outer~ sep=0
      },
      % Without units, TikZ x-coordinates are in terms of black key widths, and
      % TikZ y-coordinates are flipped (increasing goes down the page, not up).
      % This only applies to *coordinates* without units. TikZ keys like xshift
      % and yshift must have units, which greatly complicates things.
      x=\l_black_key_width_dim,
      y=-1cm
    ]
      % Label notes.
      \begin{scope}[anchor=north~ west]
        \labelCsAndConcertPitch
      \end{scope}
  }
  {
      \bool_gset_false:N \g_did_print_frequency_unit_bool

      % Label notes.
      \begin{scope}[anchor=south~ west, yshift=-\g_ambitus_y_tl cm + 0.1ex]
        \labelCsAndConcertPitch[2 - \octaveOffset]
      \end{scope}

      % Add frequencies.
      \foreach \frequency in {
        % Frequencies of A notes
        \l_min_frequency_tl, 55, 110, 220, 440, 880, 1760, 3520,
        % ISO 266 frequencies
          31.5, 40,  50,  63,  80,  100,  125,  160,  200,  250,
         315,  400, 500, 630, 800, 1000, 1250, 1600, 2000, 2500,
        3150, 4000,
        % Max frequency
        \l_max_frequency_tl
      } {
        \tl_set:Nx \l_tmpa_tl
          {
            \fp_to_decimal:n
              {
                89 * ln(\frequency / \l_min_frequency_tl) / ln(\l_max_frequency_tl / \l_min_frequency_tl) + 0.5
                % This is a simplification of
                %   (ln(\frequency) - ln(\l_min_frequency_tl)) * (89.5 - 0.5) / (ln(\l_max_frequency_tl) - ln(\l_min_frequency_tl)) + 0.5
              }
          }

        \draw[frequency-label-color, overlay] (\l_tmpa_tl, \g_ambitus_y_tl) -- ++(0, 0.05) node[anchor=north, yshift=-0.2ex] {
          \pgfmathprintnumber{\frequency}

          \bool_if:NF \g_did_print_frequency_unit_bool
            {
              \c_space_tl Hz
              \bool_gset_true:N \g_did_print_frequency_unit_bool
            }
        };
      }

      \begin{scope}[on~ background~ layer]
        % Draw white note guides.
        \tl_set:Nn \l_tmpa_tl
          {
                                   rectangle ++(1, -\g_ambitus_y_tl) % A
            ++(1, \g_ambitus_y_tl) rectangle ++(1, -\g_ambitus_y_tl) % B
            ++(2, \g_ambitus_y_tl) rectangle ++(1, -\g_ambitus_y_tl) % D
          }
        \tl_set:Nn \l_tmpb_tl { 0.05 }

        \fill[fill=white-note-guide-color]
          (0, \g_ambitus_y_tl)
          \foreach \octave in { \l_first_octave_int, ..., \l_last_octave_int } {
            \l_tmpa_tl
            ++(1,              \g_ambitus_y_tl) rectangle ++(1 - \l_tmpb_tl, -\g_ambitus_y_tl) % E
            ++(2 * \l_tmpb_tl, \g_ambitus_y_tl) rectangle ++(1 - \l_tmpb_tl, -\g_ambitus_y_tl) % F
            ++(1,              \g_ambitus_y_tl) rectangle ++(1,              -\g_ambitus_y_tl) % G
            ++(1,              \g_ambitus_y_tl)
          }
          \l_tmpa_tl;

        % Draw black note guides.
        \tl_set:Nn \l_tmpa_tl
          {
                                   rectangle ++(1, -\g_ambitus_y_tl) % A♯/B♭
            ++(2, \g_ambitus_y_tl) rectangle ++(1, -\g_ambitus_y_tl) % C♯/D♭
          }

        \fill[fill=black-note-guide-color]
          (1, \g_ambitus_y_tl)
          \foreach \octave in { \l_first_octave_int, ..., \l_last_octave_int } {
            \l_tmpa_tl
            ++(1, \g_ambitus_y_tl) rectangle ++(1, -\g_ambitus_y_tl) % D♯/E♭
            ++(2, \g_ambitus_y_tl) rectangle ++(1, -\g_ambitus_y_tl) % F♯/G♭
            ++(1, \g_ambitus_y_tl) rectangle ++(1, -\g_ambitus_y_tl) % G♯/A♭
            ++(1, \g_ambitus_y_tl)
          }
          \l_tmpa_tl;
      \end{scope}
    \end{tikzpicture}
  }

\input .note-metrics
\tl_const:Nx \l_staff_scale_factor_tl { \fp_to_decimal:n { (87 * \l_black_key_width_dim) / (\maxNoteMidpoint - \minNoteMidpoint) } }
\tl_const:Nx \l_staff_x_coordinate_tl { \fp_to_decimal:n { 0.5 - \minNoteMidpoint * \l_staff_scale_factor_tl / \l_black_key_width_dim } }

\NewDocumentCommand { \noteOctave } { m }
  { \int_eval:n { \int_div_truncate:nn { #1 } { 12 } - \octaveOffset } }

\int_const:Nn \l_first_octave_int { \noteOctave{21} }
\int_const:Nn \l_last_octave_int  { \noteOctave{\l_total_note_count_tl} }

\NewDocumentCommand { \pitchNumber } { m }
  { \int_mod:nn { #1 } { 12 } + 1 }

\seq_const_from_clist:Nn \pitch_classes { C, D♭, D, E♭, E, F, F♯, G, A♭, A, B♭, B }
\NewDocumentCommand { \pitchClass } { m }
  { \seq_item:Nn \pitch_classes { \pitchNumber{#1} } }

\NewDocumentCommand { \noteName } { m }
  { \pitchClass{#1}\noteOctave{#1} }

\NewDocumentCommand { \labelCsAndConcertPitch } { o }
  {
    \int_set:Nn \l_tmpa_int { \IfValueTF{#1}{#1}{-1} }

    \begin{scope}[
      minimum~ width=\l_black_key_width_dim,
      text=note-label-color
    ]
      \foreach \noteNumber in { 24, 36, ..., 108, 69 } {
        \tl_set:Nn \l_tmpa_tl { \noteOctave{\noteNumber} }
        \int_compare:nNnT \l_tmpa_tl > \l_tmpa_int
          { \node at (\noteNumber - 21, 0) { \pitchClass{\noteNumber}\l_tmpa_tl }; }
      }
    \end{scope}
  }

\tl_new:N    \l_long_name_tl
\tl_new:N    \l_short_name_tl
\clist_new:N \l_primary_notes_clist
\int_new:N   \l_primary_note_start_int
\int_new:N   \l_primary_note_end_int
\clist_new:N \l_primary_note_labels_clist
\tl_new:N    \l_primary_start_pitch_class_label_tl
\tl_new:N    \l_primary_end_pitch_class_label_tl
\tl_new:N    \l_primary_color_tl
\clist_new:N \l_secondary_notes_clist
\int_new:N   \l_secondary_note_start_int
\int_new:N   \l_secondary_note_end_int
\tl_new:N    \l_low_node_name_tl
\tl_new:N    \l_high_node_name_tl

\keys_define:nn { ambitus }
  {
    notes           .clist_set:N      = \l_primary_notes_clist,
    notes           .value_required:n = true,
    notes           .groups:n         = { primary-notes },
    note-labels     .clist_set:N      = \l_primary_note_labels_clist,
    note-labels     .value_required:n = true,
    color           .tl_set:N         = \l_primary_color_tl,
    color           .value_required:n = true,
    secondary-notes .clist_set:N      = \l_secondary_notes_clist,
    secondary-notes .value_required:n = true,
    low-node-name   .tl_set:N         = \l_low_node_name_tl,
    low-node-name   .value_required:n = true,
    high-node-name  .tl_set:N         = \l_high_node_name_tl,
    high-node-name  .value_required:n = true,
  }

\bool_new:N \l_secondary_note_start_is_below_primary_bool
\bool_new:N \l_secondary_note_end_is_above_primary_bool

\NewDocumentCommand { \ambitusComponent } { o o m }
  {
    \tl_clear:N    \l_long_name_tl
    \tl_clear:N    \l_short_name_tl
    \clist_clear:N \l_primary_note_labels_clist
    \tl_clear:N    \l_primary_start_pitch_class_label_tl
    \tl_clear:N    \l_primary_end_pitch_class_label_tl
    \tl_set:Nn     \l_primary_color_tl { primary-ambitus-color }
    \clist_clear:N \l_secondary_notes_clist
    \tl_clear:N    \l_low_node_name_tl
    \tl_clear:N    \l_high_node_name_tl

    \keys_set_groups:nnn { ambitus } { primary-notes } { #3 }

    \int_set:Nn \l_primary_note_start_int { \clist_item:Nn \l_primary_notes_clist 1 }
    \int_set:Nn \l_primary_note_end_int   { \clist_item:Nn \l_primary_notes_clist { -1 } }

    \keys_set_exclude_groups:nnn { ambitus } { primary-notes } { #3 }

    \int_compare:nNnT { \clist_count:N \l_primary_note_labels_clist } > 0
      {
        \tl_set:Nn \l_primary_start_pitch_class_label_tl { \clist_item:Nn \l_primary_note_labels_clist 1 }
        \tl_set:Nn \l_primary_end_pitch_class_label_tl   { \clist_item:Nn \l_primary_note_labels_clist { -1 } }
      }

    \int_case:nnF { \clist_count:N \l_secondary_notes_clist }
      {
        { 0 }
        {
          \int_set_eq:NN \l_secondary_note_start_int \l_primary_note_end_int
          \int_set_eq:NN \l_secondary_note_end_int   \l_primary_note_start_int
        }

        { 1 }
        {
          \int_compare:nNnTF { \clist_item:Nn \l_secondary_notes_clist 1 } < \l_primary_note_start_int
            {
              \int_set:Nn \l_secondary_note_start_int { \clist_item:Nn \l_secondary_notes_clist 1 }
              \int_set_eq:NN \l_secondary_note_end_int \l_primary_note_start_int
            }
            {
              \int_set_eq:NN \l_secondary_note_start_int \l_primary_note_end_int
              \int_set:Nn \l_secondary_note_end_int { \clist_item:Nn \l_secondary_notes_clist 1 }
            }
        }
      }
      {
        \int_set:Nn \l_secondary_note_start_int { \clist_item:Nn \l_secondary_notes_clist 1 }
        \int_set:Nn \l_secondary_note_end_int   { \clist_item:Nn \l_secondary_notes_clist { -1 } }
      }

    \bool_set:Nn \l_secondary_note_start_is_below_primary_bool
                 { \int_compare_p:nNn \l_secondary_note_start_int < \l_primary_note_start_int }
    \bool_set:Nn \l_secondary_note_end_is_above_primary_bool
                 { \int_compare_p:nNn \l_secondary_note_end_int > \l_primary_note_end_int }

    \IfValueT{#1}{
      \tl_set:Nn \l_long_name_tl { #1 }

      \tl_set_eq:NN \l_tmpa_tl \l_long_name_tl
      \tl_replace_all:Nnn \l_tmpa_tl { ♭ } { -flat }
      \tl_remove_all:Nn \l_tmpa_tl { . }
      \tl_remove_all:Nn \l_tmpa_tl { ( }
      \tl_remove_all:Nn \l_tmpa_tl { ) }

      \tl_if_empty:NT \l_low_node_name_tl
        { \tl_set:Nx \l_low_node_name_tl { \l_tmpa_tl \c_space_tl lowest~ note } }

      \tl_if_empty:NT \l_high_node_name_tl
        { \tl_set:Nx \l_high_node_name_tl { \l_tmpa_tl \c_space_tl highest~ note } }
    }

    \IfValueT{#2}{
      \tl_set:Nn \l_short_name_tl { #2 }
    }

    \begin{scope}[
      node~ distance=0,
      minimum~ height=\l_ambitus_bar_height_tl cm,
      xshift=-21 * \l_black_key_width_dim
    ]
      \begin{scope}[rounded~ corners=0.5bp]
        \bool_if:nT { \l_secondary_note_start_is_below_primary_bool || \l_secondary_note_end_is_above_primary_bool }
          {
            \path[fill=secondary-ambitus-color]
              (\l_secondary_note_start_int, \g_ambitus_y_tl)
              rectangle
              ++(\l_secondary_note_end_int - \l_secondary_note_start_int + 1, \l_ambitus_bar_height_tl);
          }

        \path[fill=\l_primary_color_tl]
          (\l_primary_note_start_int, \g_ambitus_y_tl)
          rectangle
          ++(\l_primary_note_end_int - \l_primary_note_start_int + 1, \l_ambitus_bar_height_tl);
      \end{scope}

      \node[note~ name] (\l_low_node_name_tl) at (\l_primary_note_start_int, \g_ambitus_y_tl) {
        \tl_if_empty:NTF \l_primary_start_pitch_class_label_tl
          { \noteName{\l_primary_note_start_int} }
          { \l_primary_start_pitch_class_label_tl \noteOctave{\l_primary_note_start_int} }
      };

      \bool_if:NT \l_secondary_note_start_is_below_primary_bool
        {
          \node[note~ name] (\l_low_node_name_tl) at (\l_secondary_note_start_int, \g_ambitus_y_tl) {
            \noteName{\l_secondary_note_start_int}
          };
        }

      \node[note~ name] (\l_high_node_name_tl) at (\l_primary_note_end_int, \g_ambitus_y_tl) {
        \int_compare:nNnT \l_primary_note_start_int < \l_primary_note_end_int
          {
            \tl_if_empty:NTF \l_primary_end_pitch_class_label_tl
              { \noteName{\l_primary_note_end_int} }
              { \l_primary_end_pitch_class_label_tl \noteOctave{\l_primary_note_end_int} }
          }
      };

      \bool_if:NT \l_secondary_note_end_is_above_primary_bool
        {
          \node[note~ name] (\l_high_node_name_tl) at (\l_secondary_note_end_int, \g_ambitus_y_tl) {
            \noteName{\l_secondary_note_end_int}
          };
        }

      \tl_if_empty:NF \l_long_name_tl
        { \node[base~ left=of~ \l_low_node_name_tl, xshift=-0.5ex, instrument~ label] { \l_long_name_tl }; }
      \tl_if_empty:NF \l_short_name_tl
        { \node[base~ right=of~ \l_high_node_name_tl, xshift=0.5ex] { \l_short_name_tl }; }
    \end{scope}
  }

\NewDocumentCommand { \ambitus } { o o m }
  {
    \ambitusComponent[#1][#2]{#3}
    \tl_gset:Nx \g_ambitus_y_tl { \fp_to_decimal:n { \g_ambitus_y_tl + (1 + \l_ambitus_bar_separation_tl) * \l_ambitus_bar_height_tl } }
  }

\NewDocumentEnvironment { instrumentFamily } { }
  { }
  { \tl_gset:Nx \g_ambitus_y_tl { \fp_to_decimal:n { \g_ambitus_y_tl + 1.5 * \l_ambitus_bar_height_tl } } }

\bool_new:N \l_ambitusComponent_did_run_bool
\NewDocumentEnvironment { bracketedAmbitus } { m }
  {
    \bool_set_false:N \l_ambitusComponent_did_run_bool
    \tl_set:Nn \l_ambitus_bar_separation_tl { 0.25 }

    \NewCommandCopy \ambitusComponentOriginal \ambitusComponent
    \RenewDocumentCommand { \ambitusComponent } { o o m }
      {
        \bool_if:NTF \l_ambitusComponent_did_run_bool
          {
            \ambitusComponentOriginal[##1][##2]{##3, low-node-name=#1~ lowest~ note}
          }
          {
            \ambitusComponentOriginal[##1][##2]{##3, high-node-name=#1~ highest~ note}
            \bool_set_true:N \l_ambitusComponent_did_run_bool
          }
      }
  }
  {
    \path[draw=ambitus-bracket-color, overlay]
      let \p1=(#1~ lowest~ note), \p2=(#1~ highest~ note) in
      (\x1 - 0.25 * \l_black_key_width_dim, \y2 + 0.5 * \l_ambitus_bar_height_tl cm) --
      (\x1 - 0.75 * \l_black_key_width_dim, \y2 + 0.5 * \l_ambitus_bar_height_tl cm) --
      (\x1 - 0.75 * \l_black_key_width_dim, \y1 - \l_ambitus_bar_height_tl cm)
      node[midway, anchor=mid~ east, xshift=-0.5ex] (#1~ midpoint) { #1 } --
      (\x1 - 0.25 * \l_black_key_width_dim, \y1 - \l_ambitus_bar_height_tl cm)
      % let \p3=(#1~ midpoint) in
      % (\x1 - 0.75 * \l_black_key_width_dim, \y3) -- (\x1 - \l_black_key_width_dim, \y3)
      ;

    \RenewCommandCopy \ambitusComponent \ambitusComponentOriginal

    \tl_gset:Nx \g_ambitus_y_tl { \fp_to_decimal:n { \g_ambitus_y_tl + \l_ambitus_bar_height_tl } }
  }

\NewDocumentEnvironment { compactAmbitus } { }
  { \tl_set:Nn \l_ambitus_bar_separation_tl { 0 } }
  { }

\tl_const:Nx \l_keyboard_height_tl { \fp_to_decimal:n { 6 * \l_ambitus_bar_height_tl } }
\tl_const:Nx \l_black_key_height_tl { \fp_to_decimal:n { 0.625 * \l_keyboard_height_tl } }
\tl_const:Nx \l_white_to_black_key_distance_tl { \fp_to_decimal:n { \l_keyboard_height_tl - \l_black_key_height_tl } }
\NewDocumentCommand { \keyboard } { }
  {
    \tl_gset:Nx \g_ambitus_y_tl { \fp_to_decimal:n { \g_ambitus_y_tl - (0.5 + \l_ambitus_bar_separation_tl) * \l_ambitus_bar_height_tl } }
    \begin{scope}[yshift=-\g_ambitus_y_tl cm]
      \begin{scope}[draw=black-key-color]
        \draw[fill=white] (0, 0) rectangle (\l_total_note_count_tl, \l_keyboard_height_tl);

        \tl_set:Nn \l_tmpa_tl
          {
            ++( 1.5, \l_keyboard_height_tl) --        ++(0, -\l_white_to_black_key_distance_tl)
            ++(-0.5, 0)                     rectangle ++(1, -\l_black_key_height_tl) % A♯/B♭
            ++( 1  , \l_keyboard_height_tl) --        ++(0, -\l_keyboard_height_tl)
            ++( 1.5, \l_keyboard_height_tl) --        ++(0, -\l_white_to_black_key_distance_tl)
            ++(-0.5, 0)                     rectangle ++(1, -\l_black_key_height_tl) % C♯/D♭
          }

        \filldraw[fill=black-key-color]
          (0, 0)
          \foreach \octave in { \l_first_octave_int, ..., \l_last_octave_int } {
            \l_tmpa_tl
            ++( 1.5, \l_keyboard_height_tl) --        ++(0, -\l_white_to_black_key_distance_tl)
            ++(-0.5, 0)                     rectangle ++(1, -\l_black_key_height_tl) % D♯/E♭
            ++( 1  , \l_keyboard_height_tl) --        ++(0, -\l_keyboard_height_tl)
            ++( 1.5, \l_keyboard_height_tl) --        ++(0, -\l_white_to_black_key_distance_tl)
            ++(-0.5, 0)                     rectangle ++(1, -\l_black_key_height_tl) % F♯/G♭
            ++( 1.5, \l_keyboard_height_tl) --        ++(0, -\l_white_to_black_key_distance_tl)
            ++(-0.5, 0)                     rectangle ++(1, -\l_black_key_height_tl) % G♯/A♭
          }
          \l_tmpa_tl;

        % Label notes.
        \begin{scope}[anchor=north~ west, yshift=-0.3ex]
          \labelCsAndConcertPitch
        \end{scope}
      \end{scope}

      % Add MIDI note numbers.
      \begin{scope}[midi~ note~ number]
        \foreach \noteNumber in { 21, ..., 110 } {
          \int_case:nnTF { \pitchNumber{\noteNumber} }
            { { 2 } { } { 4 } { } { 7 } { } { 9 } { } { 11 } { } }
            { \tl_set:Nn \l_tmpa_tl { black-note-number-color } }
            { \tl_set:Nn \l_tmpa_tl { white-note-number-color } }
          \node[text=\l_tmpa_tl] at (\noteNumber - 21, 0.5 * \l_keyboard_height_tl) { \noteNumber };
        }
      \end{scope}

      \node[
        align=center,
        anchor=north~ east,
        inner~ xsep=1pt,
        minimum~ height=\l_keyboard_height_tl cm,
        overlay,
        text=white-note-number-color
      ] (MIDI~ note~ label) at (0, 0) { MIDI\\[-0.5ex] note~ nos. };
    \end{scope}

    \node[below=0~ of~ MIDI~ note~ label, overlay] { };
    \pgfgetlastxy{\l_tmpa_dim}{\l_tmpb_dim}
    \tl_gset:Nx \g_ambitus_y_tl { \fp_to_decimal:n { \dim_to_decimal_in_cm:n { -\l_tmpb_dim } + \l_ambitus_bar_height_tl } }
  }

\NewDocumentCommand { \staff } { m O{} }
  {
    \pgfgetlastxy{\l_tmpa_dim}{\l_tmpb_dim}
    \begin{scope}[#2]
      \node[anchor=north~ west, overlay] (#1~ staff) at (\l_staff_x_coordinate_tl, \l_tmpb_dim) {
        \includegraphics[scale=\l_staff_scale_factor_tl]{notes-#1.pdf}
      };
    \end{scope}

    \node[below=0~ of~ #1~ staff] { };
    \pgfgetlastxy{\l_tmpa_dim}{\l_tmpb_dim}
    \tl_gset:Nx \g_ambitus_y_tl { \dim_to_decimal_in_cm:n { -\l_tmpb_dim } }
  }

\ExplSyntaxOff


\begin{document}

\section*{Instrumentation Cheat Sheet}
\vspace{2ex}

\begin{instrumentationPicture}
  % Add the top staff.
  \ExplSyntaxOn
  \staff{top}[yshift=-\l_ambitus_bar_height_tl cm]
  \tl_gset:Nx \g_ambitus_y_tl { \fp_to_decimal:n { \g_ambitus_y_tl - 2 * \l_ambitus_bar_height_tl } }
  \ExplSyntaxOff

  % Add wind instrument ambitus.
  \begin{instrumentFamily}
    \ambitus  [Piccolo]             [Picc.]   {notes={74, 108}}
    \ambitus  [Flute]               [Fl.]     {notes={60,  98}, secondary-notes=59}
    \ambitus  [Alto Flute]          [A.Fl.]   {notes={55,  88}}
    \ambitus  [Oboe]                [Ob.]     {notes={58,  91}, secondary-notes=96}
    \ambitus  [English Horn]        [EH]      {notes={52,  83}}
    \ambitus  [E♭ Clarinet]         [E♭ Cl.]  {notes={55,  95}}
    \ambitus  [B♭ Clarinet]         [B♭ Cl.]  {notes={50,  91}}
    \ambitus  [Bass Clarinet]       [B.Cl.]   {notes={34,  79}, secondary-notes=83}
    \ambitus  [Bassoon]             [Bsn.]    {notes={34,  75}, secondary-notes=77}
    \begin{scope}[instrument label/.append style={anchor=north west, xshift=0.5ex, yshift=-0.3ex}]
      \ambitus[Contrabassoon]       [Cbsn.]   {notes={22,  60}, secondary-notes=21}
    \end{scope}
  \end{instrumentFamily}

  \begin{instrumentFamily}
    \ambitus[Soprano Saxophone]  [S. Sax.]   {notes={56, 94}}
    \ambitus[Alto Saxophone]     [A. Sax.]   {notes={49, 87}}
    \ambitus[Tenor Saxophone]    [T. Sax.]   {notes={44, 82}}
    \ambitus[Baritone Saxophone] [Bar. Sax.] {notes={37, 75}, secondary-notes=36}

    \ExplSyntaxOn
    \node[
      annotation,
      anchor=mid~ west,
      right=3.5 * \l_black_key_width_dim~ of~ Soprano~ Saxophone~ highest~ note,
      text~ width=12.5 * \l_black_key_width_dim
    ] {
      Higher~ notes~ are~ possible~ on~ wind~ instruments,~ but~ increasingly~ difficult~ to~ sound~ and~ not~ always~ available~ in~ sample~ sets.
    };
    \ExplSyntaxOff
  \end{instrumentFamily}

  \begin{instrumentFamily}
    \ambitus[French Horn]        [Hn.]           {notes={31, 77}}
    \ambitus[B♭ Piccolo Trumpet] [B♭ Picc. Tpt.] {notes={64, 91}, secondary-notes=62}
    \ambitus[C Trumpet]          [C Tpt.]        {notes={54, 84}, secondary-notes=53}
    \ambitus[B♭ Trumpet]         [B♭ Tpt.]       {notes={52, 82}, secondary-notes=51}
    \ambitus[Trombone]           [Tbn.]          {notes={40, 77}}
    \ambitus[Bass Trombone]      [B. Tbn.]       {notes={34, 77}}
    \ambitus[C Tuba]             [C Tuba]        {notes={26, 64}, secondary-notes=24}
    \ambitus[B♭ Tuba]            [B♭ Tuba]       {notes={22, 63}}

    \ExplSyntaxOn
    \path
      let \p1=(C~ Trumpet~ lowest~ note) in
      (0, \y1)
      node[
        annotation,
        anchor=north~ west,
        text~ width=7.5 * \l_black_key_width_dim
      ] {
        Lower~ pedal~ tones~ are~ possible~ on~ brass~ instruments.
      };
    \ExplSyntaxOff
  \end{instrumentFamily}

  % Draw the keyboard.
  \keyboard

  % Add percussion and string instrument ambitus.
  \begin{instrumentFamily}
    \begin{bracketedAmbitus}{Timpani}
      \begin{compactAmbitus}
        \ambitus{notes={53, 61}, color=secondary-ambitus-color}
        \ambitus{notes={50, 57}}
        \ambitus{notes={46, 53}}
        \ambitus{notes={41, 48}}
        \ambitus{notes={36, 45}, color=secondary-ambitus-color}
      \end{compactAmbitus}
    \end{bracketedAmbitus}

    % Move high pitched percussion up to save space.
    \ExplSyntaxOn
    \node[below=0~ of~ MIDI~ note~ label, overlay] { };
    \pgfgetlastxy{\l_tmpa_dim}{\l_tmpb_dim}
    \tl_gset:Nx \g_ambitus_y_tl { \fp_to_decimal:n { \dim_to_decimal_in_cm:n { -\l_tmpb_dim } + 2 * \l_ambitus_bar_height_tl } }
    \ExplSyntaxOff

    \ambitus[Crotales]      [Crot.]    {notes={84, 108}}
    \ambitus[Glockenspiel]  [Glck.]    {notes={79, 108}, secondary-notes=72}
    \ambitus[Celesta]       [Cel.]     {notes={60, 108}}
    \ambitus[Tubular Bells] [T. Bells] {notes={60,  77}, secondary-notes={52, 79}}
    \ambitus[Vibraphone]    [Vib.]     {notes={53,  89}, secondary-notes=48}
    \ambitus[Xylophone]     [Xyl.]     {notes={65, 108}, secondary-notes=58}
    \ambitus[Marimba]       [Mar.]     {notes={48,  96}, secondary-notes=36}
    \ambitus[Harp]          [Harp]     {notes={23, 104}, note-labels={C♭, G♯}}
    \ambitus[Piano]         [Pno.]     {notes={21, 108}}
  \end{instrumentFamily}

  \begin{instrumentFamily}
    \begin{bracketedAmbitus}{Violin}
      \ambitusComponent[Vln. harmonics]{notes=67} \ambitusComponent{notes=74} \ambitus{notes={79, 108}, secondary-notes=110}
      \begin{compactAmbitus}
        \ambitus{notes={76, 100}}
        \ambitus{notes={69,  93}}
        \ambitus{notes={62,  86}}
        \ambitus{notes={55,  79}}
      \end{compactAmbitus}
    \end{bracketedAmbitus}

    \begin{bracketedAmbitus}{Viola}
      \ambitusComponent[Vla. harmonics]{notes=60} \ambitusComponent{notes=67} \ambitus{notes={72, 100}, secondary-notes=101}
      \begin{compactAmbitus}
        \ambitus{notes={69, 93}}
        \ambitus{notes={62, 86}}
        \ambitus{notes={55, 79}}
        \ambitus{notes={48, 72}}
      \end{compactAmbitus}
    \end{bracketedAmbitus}

    \begin{bracketedAmbitus}{Cello}
      \ambitusComponent[Vcl. harmonics]{notes=48} \ambitusComponent{notes=55} \ambitus{notes={60, 89}, secondary-notes=95}
      \begin{compactAmbitus}
        \ambitus{notes={57, 81}, secondary-notes=82}
        \ambitus{notes={50, 67}, secondary-notes=74}
        \ambitus{notes={43, 60}, secondary-notes=67}
        \ambitus{notes={36, 53}, secondary-notes=60}
      \end{compactAmbitus}

      \ExplSyntaxOn
      \node[
        annotation,
        anchor=mid~ west,
        right=47.5 * \l_black_key_width_dim~ of~ Cello~ highest~ note,
        text~ width=14.5 * \l_black_key_width_dim
      ] {
        Higher~ harmonics~ are~ possible~ on~ stringed~ instruments,~ but~ not~ always~ available~ in~ sample~ sets.
      };
      \ExplSyntaxOff
    \end{bracketedAmbitus}

    \begin{bracketedAmbitus}{Contrabass}
      \ambitusComponent[Cb. harmonics]{notes=40} \ambitusComponent{notes=45} \ambitusComponent{notes=47} \ambitusComponent{notes=50} \ambitus{notes={52, 84}, secondary-notes=89}
      \begin{compactAmbitus}
        \ambitus{notes={43, 55}, secondary-notes=67}
        \ambitus{notes={38, 50}, secondary-notes=62}
        \ambitus{notes={33, 45}, secondary-notes=57}
        \ambitus{notes={28, 40}, secondary-notes={24, 52}}
      \end{compactAmbitus}
    \end{bracketedAmbitus}
  \end{instrumentFamily}

  % Add the bottom staff.
  \ExplSyntaxOn
  \staff{bottom}[yshift=1.75 * \l_ambitus_bar_height_tl cm]
  \ExplSyntaxOff
\end{instrumentationPicture}

\clearpage

\begin{instrumentationPicture}
  \ExplSyntaxOn
  \tl_gset:Nx \g_ambitus_y_tl { \fp_to_decimal:n { 1.5 * \l_ambitus_bar_height_tl } }
  \ExplSyntaxOff

  % Add wind instrument ambitus.
  \begin{instrumentFamily}
    \ambitus  [Heckelphone]         [Heck.]   {notes={45,  79}, secondary-notes=84}
    \ambitus  [E♭ Alto Clarinet]    [A.Cl.]   {notes={43,  79}, secondary-notes={42, 84}}
    \begin{scope}[instrument label/.append style={anchor=north west, xshift=0.5ex, yshift=-0.3ex}]
      \ambitus[Contrabass Clarinet] [Cb. Cl.] {notes={22,  67}}
    \end{scope}
  \end{instrumentFamily}

  \begin{instrumentFamily}
    \ambitus[B♭ Bass Trumpet] [B♭ B. Tpt.] {notes={40, 72}}
    \ambitus[B♭ Wagner Tuba]  [B♭ Wag.]    {notes={40, 77}}
    \ambitus[F Wagner Tuba]   [F Wag.]     {notes={35, 77}}
    \ambitus[B♭ Baritone]     [B♭ Bar.]    {notes={40, 72}}
    \ambitus[B♭ Euphonium]    [B♭ Euph.]   {notes={35, 72}}
  \end{instrumentFamily}

  % Draw the keyboard.
  \keyboard

  % Add the staff.
  \ExplSyntaxOn
  \staff{top}[yshift=-0.5 * \l_ambitus_bar_height_tl cm]
  \ExplSyntaxOff
\end{instrumentationPicture}

\begin{multicols*}{4}
  \raggedcolumns
  \scriptsize
  \addfontfeature{Numbers=OldStyle}

  \section*{Instrument Transpositions}

  \begin{tabular}{@{}lll@{}}
    \toprule
    Instrument          & To Concert Pitch        \\
    \midrule
    Piccolo             & octave up               \\
    Alto Flute          & 4th down                \\
    English Horn        & 5th down                \\
    Heckelphone         & octave down             \\
    E♭ Clarinet         & minor~3rd up            \\
    B♭ Clarinet         & 2nd down                \\
    E♭ Alto Clarinet    & minor 6th down          \\
    Bass Clarinet       & 2nd + octave down       \\
    Contrabass Clarinet & 2nd + 2 octaves down    \\
    Contrabassoon       & octave down             \\
    \midrule
    Soprano Saxophone   & 2nd down                \\
    Alto Saxophone      & minor 6th down          \\
    Tenor Saxophone     & 2nd + octave down       \\
    Baritone Saxophone  & minor 6th + octave down \\
    \midrule
    French Horn in F    & \begin{tabular}{@{}l@{}}
                            5th down, treble \& new bass clef \\
                            4th up, old bass clef
                          \end{tabular}           \\
    B♭ Piccolo Trumpet  & minor 7th up            \\
    B♭ Trumpet          & 2nd down                \\
    B♭ Bass Trumpet     & 2nd + octave down       \\
    B♭ Wagner Tuba      & 2nd down                \\
    F Wagner Tuba       & 5th down                \\
    B♭ Baritone         & 2nd + octave down       \\
    B♭ Euphonium        & 2nd + octave down       \\
    \midrule
    Crotales            & 2 octaves up            \\
    Glockenspiel        & 2 octaves up            \\
    Celesta             & octave up               \\
    Xylophone           & octave up               \\
    \midrule
    Contrabass          & octave down             \\
    \bottomrule
  \end{tabular}


  \section*{Note Names}

  \def\rolandOctaves{“Roland” octaves used in Ableton Live, Kontakt, and Pro Tools (concert pitch is {\addfontfeature{Numbers=Lining}A3})}
  \def\yamahaOctaves{“Yamaha” octaves used in FabFilter {\addfontfeature{Numbers=Lining}Pro‑Q} and Logic’s default (concert pitch is {\addfontfeature{Numbers=Lining}A4})}

  Note names have
  \ExplSyntaxOn
  \msg_new:nnn { instrumentation-cheat-sheet } { invalid-octave-offset }
    { \c_backslash_str octaveOffset~ is~ #1~ but~ should~ be~ 1~ or~ 2. }
  \int_case:nnF \octaveOffset
    {
      { 1 }
      { \yamahaOctaves.~ Subtract~ 1~ for~ \rolandOctaves. }

      { 2 }
      { \rolandOctaves.~ Add~ 1~ for~ \yamahaOctaves. }
    }
    { \msg_error:nne { instrumentation-cheat-sheet } { invalid-octave-offset } \octaveOffset }
  \ExplSyntaxOff


  \columnbreak
  \section*{Page Numbers of References}

  \newcolumntype{R}{>{\addfontfeature{Numbers={Lining,Monospaced}}}r}

  \def\thinrule{\specialrule{\cmidrulewidth}{0.5\aboverulesep}{0.5\belowrulesep}}

  \newlength\instrumentWidth
  \settowidth{\instrumentWidth}{Contrabass Clarinet}

  \addtolength{\tabcolsep}{-0.4em}

  % A longtable environment would be preferred, but longtable can’t be used in a
  % multicols environment.
  \begin{tabular}[t]{@{}p{\instrumentWidth}RRRR@{}}
    \toprule
                        & \multicolumn{4}{c}{Page Numbers} \\
                          \cmidrule(){2-5}
    Instrument          & \citeauthor{adler}    & \citeauthor{blatter}        & \citeauthor{forsyth}    & \citeauthor{sevsay}    \\
    \midrule
    Piccolo             & \notecite[198]{adler} & \multirow{3}{*}{
                                                  \notecite[90]{blatter}}     & \notecite[198]{forsyth} & \notecite[76]{sevsay}  \\
    Flute               & \notecite[189]{adler} &                             & \notecite[182]{forsyth} & \notecite[75]{sevsay}  \\
    Alto Flute          & \notecite[201]{adler} &                             & \notecite[196]{forsyth} & \notecite[77]{sevsay}  \\
    \thinrule
    Oboe                & \notecite[204]{adler} & \multirow{3}{*}{
                                                  \notecite[98]{blatter}}     & \notecite[204]{forsyth} & \notecite[78]{sevsay}  \\
    English Horn        & \notecite[209]{adler} &                             & \notecite[220]{forsyth} & \notecite[79]{sevsay}  \\
    Heckelphone         & \notecite[216]{adler} &                             & \notecite[228]{forsyth} & \notecite[80]{sevsay}  \\
    \thinrule
    E♭ Clarinet         & \notecite[224]{adler} & \multirow{5}{*}{
                                                  \notecite[105]{blatter}}    & \notecite[278]{forsyth} & \notecite[82]{sevsay}  \\
    B♭ Clarinet         & \notecite[217]{adler} &                             & \notecite[251]{forsyth} & \notecite[80]{sevsay}  \\
    E♭ Alto Clarinet    & \notecite[228]{adler} &                             & \notecite[282]{forsyth} & \notecite[83]{sevsay}  \\
    Bass Clarinet       & \notecite[225]{adler} &                             & \notecite[272]{forsyth} & \notecite[82]{sevsay}  \\
    Contrabass Clarinet & \notecite[230]{adler} &                             & \notecite[286]{forsyth} & \notecite[84]{sevsay}  \\
    \thinrule
    Bassoon             & \notecite[235]{adler} & \multirow{2}{*}{
                                                  \notecite[116]{blatter}}    & \notecite[229]{forsyth} & \notecite[84]{sevsay}  \\
    Contrabassoon       & \notecite[240]{adler} &                             & \notecite[246]{forsyth} & \notecite[85]{sevsay}  \\
    \midrule
    Saxophones          & \notecite[231]{adler} & \notecite[126]{blatter}     & \notecite[166]{forsyth} & \notecite[87]{sevsay}  \\
    \midrule
    French Horn         & \notecite[337]{adler} & \notecite[148]{blatter}     & \notecite[109]{forsyth} & \notecite[96]{sevsay}  \\
    Trumpets            & \notecite[357]{adler} & \notecite[159]{blatter}     & \notecite[89]{forsyth}  & \notecite[101]{sevsay} \\
    Trombones           & \notecite[368]{adler} & \notecite[169]{blatter}     & \notecite[133]{forsyth} & \notecite[103]{sevsay} \\
    Wagner Tubas        & —                     & \notecite[148]{blatter}     & \notecite[153]{forsyth} & \notecite[111]{sevsay} \\
    Tubas               & \notecite[377]{adler} & \notecite[178]{blatter}     & \notecite[151]{forsyth} & \notecite[108]{sevsay} \\
    \midrule
    Timpani             & \notecite[485]{adler} & \notecite[209]{blatter}     & \notecite[41]{forsyth}  & \notecite[167]{sevsay} \\
    \midrule
    Crotales            & \notecite[481]{adler} & \notecite[206]{blatter}     & —                       & \notecite[161]{sevsay} \\
    Glockenspiel        & \notecite[479]{adler} & \notecite[205]{blatter}     & \notecite[60]{forsyth}  & \notecite[150]{sevsay} \\
    Tubular Bells       & \notecite[480]{adler} & \notecite[205]{blatter}     & \notecite[53]{forsyth}  & \notecite[155]{sevsay} \\
    Celesta             & \notecite[528]{adler} & \notecite[206]{blatter}     & \notecite[64]{forsyth}  & \notecite[245]{sevsay} \\
    Vibraphone          & \notecite[477]{adler} & \notecite[205]{blatter}     & —                       & \notecite[152]{sevsay} \\
    Xylophone           & \notecite[475]{adler} & \notecite[204]{blatter}     & \notecite[66]{forsyth}  & \notecite[147]{sevsay} \\
    Marimba             & \notecite[476]{adler} & \notecite[204]{blatter}     & —                       & \notecite[149]{sevsay} \\
    \midrule
    Harp                & \notecite[95]{adler}  & \notecite[252]{blatter}     & \notecite[461]{forsyth} & \notecite[194]{sevsay} \\
    Piano               & \notecite[521]{adler} & \notecite[242]{blatter}     & —                       & \notecite[230]{sevsay} \\
    \midrule
    Violin              & \notecite[57]{adler}  & \notecite[49, 441]{blatter} & \notecite[303]{forsyth} & \multirow{4}{*}{
                                                                                                          \notecite[3]{sevsay}}  \\
    Viola               & \notecite[71]{adler}  & \notecite[56, 442]{blatter} & \notecite[381]{forsyth} &                        \\
    Cello               & \notecite[81]{adler}  & \notecite[60, 443]{blatter} & \notecite[409]{forsyth} &                        \\
    Contrabass          & \notecite[89]{adler}  & \notecite[67, 444]{blatter} & \notecite[436]{forsyth} &                        \\
    \bottomrule
  \end{tabular}

  \nocite{vienna-academy}

  \renewcommand*\bibfont{\scriptsize}
  \setlength\bibhang{10pt}
  \setlength\bibitemsep{\parskip}
  \printbibliography


  \section*{Legal}

  % https://www.ableton.com/en/legal/trademark-list/
  Ableton and Live are trademarks of Ableton AG.

  % https://www.apple.com/legal/intellectual-property/trademark/appletmlist.html
  Logic is a trademark of Apple Inc., registered in the U.S. and other countries and regions.

  % https://www.avid.com/legal/trademarks-and-other-notices
  Pro Tools is a registered trademark of Avid Technology, Inc.

  % https://tsdr.uspto.gov/#caseNumber=79394860&caseSearchType=US_APPLICATION&caseType=DEFAULT&searchType=statusSearch
  FabFilter is a trademark of FabFilter BV.

  % https://tsdr.uspto.gov/#caseNumber=73083201&caseSearchType=US_APPLICATION&caseType=DEFAULT&searchType=statusSearch
  Roland is a registered trademark of Roland Corporation.

  % https://tsdr.uspto.gov/#caseNumber=77642839&caseSearchType=US_APPLICATION&caseType=DEFAULT&searchType=statusSearch
  Vienna Symphonic Library is a registered trademark of Vienna Symphonic Library GmbH.

  % https://www.yamaha.com/paragon/trademarkguidelines/
  Yamaha is a registered trademark of Yamaha Corporation.
\end{multicols*}

\end{document}
