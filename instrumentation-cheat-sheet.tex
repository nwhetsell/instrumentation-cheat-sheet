\documentclass{article}

\usepackage[american]{babel}
\hyphenation{
Fab-Fil-ter
}
\usepackage{csquotes}

\usepackage[
  landscape,
  margin=0.5in,
  footskip=0.25in
]{geometry}
\parindent=0sp
\parskip=0.125cm
\topskip=1sp

\usepackage[
  backend=biber,
  isbn=false
]{biblatex-chicago}
\bibliography{instrumentation-cheat-sheet}

\usepackage{array}
\usepackage{booktabs}
\usepackage{datetime2}
\usepackage{fancyhdr}
\usepackage{multicol}
\usepackage{multirow}

\usepackage[no-config]{fontspec}
\setmainfont{Libertinus Sans}[Numbers=Proportional]
\setmonofont{TeX Gyre Cursor}
\newfontfamily\symbolFont{Libertinus Serif}

\usepackage{newunicodechar}
\newunicodechar{♯}{{\symbolFont\char`\♯}}
\newunicodechar{♭}{\hspace{-0.15ex}{\symbolFont\char`\♭}\hspace{-0.15ex}}

\usepackage{tikz}
\usetikzlibrary{backgrounds}
\usetikzlibrary{calc}
\usetikzlibrary{positioning}

\definecolor{primary-ambitus-color}  {gray}{0.55}
\definecolor{secondary-ambitus-color}{gray}{0.72}
\definecolor{ambitus-bracket-color}  {gray}{0.5}
\definecolor{white-note-guide-color} {gray}{0.975}
\definecolor{black-key-color}        {gray}{0.3}
\definecolor{black-note-guide-color} {gray}{0.93}
\definecolor{note-label-color}       {gray}{0.7}
\definecolor{frequency-label-color}  {gray}{0.7}
\definecolor{white-note-number-color}{gray}{0.8}
\definecolor{black-note-number-color}{gray}{0.525}

\tikzset{annotation/.style={font=\itshape, text badly ragged}}
\tikzset{instrument label/.style={overlay}}

\input .octave-offset.tex

\usepackage[
  pdftitle={Instrumentation Cheat Sheet},
  pdfauthor={Nathan Whetsell},
  hidelinks
]{hyperref}

\makeatletter
\renewcommand\section{\@startsection {section}{1}{\z@}%
                                     {-2ex \@plus -1ex \@minus -.2ex}%
                                     {1sp \@plus .2ex}%
                                     {\normalfont\normalsize}}%
\makeatother

\begin{document}
\frenchspacing
\pagenumbering{gobble}
\pagestyle{fancyplain}
\fancyhf{}
\renewcommand{\headrulewidth}{0sp}
\renewcommand{\footrulewidth}{0sp}
\lfoot{\fancyplain{}{\tiny commit \ttfamily \input|"git rev-parse --short HEAD"}}
\rfoot{\fancyplain{}{\tiny\DTMsetstyle{iso}\today}}

\section*{Instrumentation Cheat Sheet}
\vfil

\def\totalNoteCount{90}
\pgfmathsetmacro{\blackKeyWidth}{\textwidth / \totalNoteCount}

\begin{tikzpicture}[
  every node/.style={
    node font=\tiny,
    inner sep=0,
    outer sep=0
  },
  x=\blackKeyWidth,
  y=-1 cm
]
  \def\ambitusBarHeight{0.155}
  \def\ambitusBarSeparation{0.45}
  \pgfmathsetmacro{\keyboardHeight}{6 * \ambitusBarHeight}
  \pgfmathsetmacro{\labelOffset}{-0.1 * \blackKeyWidth}

  \input .note-metrics
  \pgfmathsetmacro{\noteImageScaleFactor}{(87 * \blackKeyWidth) / (\maxNoteMidpoint - \minNoteMidpoint)}
  \pgfmathsetmacro{\noteImageXCoordinate}{0.5 - \minNoteMidpoint * \noteImageScaleFactor / \blackKeyWidth}

  \ExplSyntaxOn

  \int_new:N \l_labeled_note_minimum_octave_int

  \NewDocumentCommand { \labelCsAndConcertPitch } { o } {
    \IfValueTF{#1}{
      \int_set:Nn \l_labeled_note_minimum_octave_int { #1 }
    } {
      \int_set:Nn \l_labeled_note_minimum_octave_int { -1 }
    }
    \begin{scope}[
      minimum~ width=\blackKeyWidth,
      text=note-label-color
    ]
      \foreach \noteNumber in { 24, 36, ..., 108, 69 } {
        \pgfmathsetmacro{\octave}{int(div(\noteNumber, 12) - \octaveOffset)}
        \bool_if:nT { \int_compare_p:nNn \octave > \l_labeled_note_minimum_octave_int } {
          \node (midi-note-\noteNumber) at (\noteNumber - 21, 0) {
            \bool_if:nTF { \int_compare_p:nNn \noteNumber = { 69 } } { A } { C } \octave
          };
        }
      }
    \end{scope}
  }

  \NewDocumentCommand { \noteOctave } { m } {
    \int_eval:n { \int_div_truncate:nn { #1 } { 12 } - \octaveOffset }
  }

  \seq_const_from_clist:Nn \note_names { C, D♭, D, E♭, E, F, F♯, G, A♭, A, B♭, B }
  \NewDocumentCommand { \noteName } { m } {
    \seq_item:Nn \note_names { \int_mod:nn { #1 } { 12 } + 1 }
    \int_eval:n { \int_div_truncate:nn { #1 } { 12 } - \octaveOffset }
  }

  \tl_new:N    \l_long_name_tl
  \tl_new:N    \l_short_name_tl
  \clist_new:N \l_primary_notes_clist
  \int_new:N   \l_primary_note_start_int
  \int_new:N   \l_primary_note_end_int
  \clist_new:N \l_primary_note_labels_clist
  \tl_new:N    \l_primary_start_pitch_class_label_tl
  \tl_new:N    \l_primary_end_pitch_class_label_tl
  \tl_new:N    \l_primary_color_tl
  \clist_new:N \l_secondary_notes_clist
  \int_new:N   \l_secondary_note_start_int
  \int_new:N   \l_secondary_note_end_int
  \tl_new:N    \l_low_node_name_tl
  \tl_new:N    \l_high_node_name_tl

  \keys_define:nn { ambitus } {
    notes           .clist_set:N      = \l_primary_notes_clist,
    notes           .value_required:n = true,
    notes           .groups:n         = { primary-notes },
    note-labels     .clist_set:N      = \l_primary_note_labels_clist,
    note-labels     .value_required:n = true,
    color           .tl_set:N         = \l_primary_color_tl,
    color           .value_required:n = true,
    secondary-notes .clist_set:N      = \l_secondary_notes_clist,
    secondary-notes .value_required:n = true,
    low-node-name   .tl_set:N         = \l_low_node_name_tl,
    low-node-name   .value_required:n = true,
    high-node-name  .tl_set:N         = \l_high_node_name_tl,
    high-node-name  .value_required:n = true,
  }

  \bool_new:N \l_secondary_note_start_is_below_primary_bool
  \bool_new:N \l_secondary_note_end_is_above_primary_bool

  \NewDocumentCommand { \ambitusComponent } { o o m } {
    \tl_clear:N    \l_long_name_tl
    \tl_clear:N    \l_short_name_tl
    \clist_clear:N \l_primary_note_labels_clist
    \tl_clear:N    \l_primary_start_pitch_class_label_tl
    \tl_clear:N    \l_primary_end_pitch_class_label_tl
    \tl_set:Nn     \l_primary_color_tl { primary-ambitus-color }
    \clist_clear:N \l_secondary_notes_clist
    \tl_clear:N    \l_low_node_name_tl
    \tl_clear:N    \l_high_node_name_tl

    \keys_set_groups:nnn { ambitus } { primary-notes } { #3 }

    \int_set:Nn \l_primary_note_start_int { \clist_item:Nn \l_primary_notes_clist 1 }
    \int_set:Nn \l_primary_note_end_int   { \clist_item:Nn \l_primary_notes_clist { -1 } }

    \keys_set_exclude_groups:nnn { ambitus } { primary-notes } { #3 }

    % \tl_show:N \l_low_node_name_tl
    % \tl_show:N \l_high_node_name_tl

    \bool_if:nT { \int_compare_p:nNn { \clist_count:N \l_primary_note_labels_clist } > 0 } {
      \tl_set:Nn \l_primary_start_pitch_class_label_tl { \clist_item:Nn \l_primary_note_labels_clist 1 }
      \tl_set:Nn \l_primary_end_pitch_class_label_tl   { \clist_item:Nn \l_primary_note_labels_clist { -1 } }
    }

    \bool_case:nF {
      { \int_compare_p:nNn { \clist_count:N \l_secondary_notes_clist } = 0 } {
        \int_set_eq:NN \l_secondary_note_start_int \l_primary_note_end_int
        \int_set_eq:NN \l_secondary_note_end_int   \l_primary_note_start_int
      }
      { \int_compare_p:nNn { \clist_count:N \l_secondary_notes_clist } = 1 } {
        \bool_if:nTF { \int_compare_p:nNn { \clist_item:Nn \l_secondary_notes_clist 1 } < \l_primary_note_start_int } {
          \int_set:Nn \l_secondary_note_start_int { \clist_item:Nn \l_secondary_notes_clist 1 }
          \int_set_eq:NN \l_secondary_note_end_int \l_primary_note_start_int
        } {
          \int_set_eq:NN \l_secondary_note_start_int \l_primary_note_end_int
          \int_set:Nn \l_secondary_note_end_int { \clist_item:Nn \l_secondary_notes_clist 1 }
        }
      }
    } {
      \int_set:Nn \l_secondary_note_start_int { \clist_item:Nn \l_secondary_notes_clist 1 }
      \int_set:Nn \l_secondary_note_end_int   { \clist_item:Nn \l_secondary_notes_clist { -1 } }
    }

    \IfValueT{#1}{
      \tl_set:Nn \l_long_name_tl { #1 }

      \tl_set_eq:NN \l_tmpa_tl \l_long_name_tl
      \tl_replace_all:Nnn \l_tmpa_tl { ♭ } { -flat }
      \tl_remove_all:Nn \l_tmpa_tl { . }
      \tl_remove_all:Nn \l_tmpa_tl { ( }
      \tl_remove_all:Nn \l_tmpa_tl { ) }

      \tl_if_empty:NT \l_low_node_name_tl {
        \tl_set:Nn \l_low_node_name_tl { \l_tmpa_tl-low }
      }
      \tl_if_empty:NT \l_high_node_name_tl {
        \tl_set:Nn \l_high_node_name_tl { \l_tmpa_tl-high }
      }
    }

    \IfValueT{#2}{
      \tl_set:Nn \l_short_name_tl { #2 }
    }

    \begin{scope}[node~ distance=0]
      \pgfmathsetmacro{\ambitusMinY}{\ambitusBarOffset * \ambitusBarHeight}

      \begin{scope}[rounded~ corners=0.5bp]
        \bool_set:Nn \l_secondary_note_start_is_below_primary_bool
                     { \int_compare_p:nNn { \l_secondary_note_start_int } < { \l_primary_note_start_int } }
        \bool_set:Nn \l_secondary_note_end_is_above_primary_bool
                     { \int_compare_p:nNn { \l_secondary_note_end_int } > { \l_primary_note_end_int } }

        \bool_if:nT { \l_secondary_note_start_is_below_primary_bool || \l_secondary_note_end_is_above_primary_bool } {
          \path[fill=secondary-ambitus-color]
            (\l_secondary_note_start_int - 21, \ambitusMinY)
            rectangle
            ++(\l_secondary_note_end_int - \l_secondary_note_start_int + 1, \ambitusBarHeight);
        }

        \path[fill=\l_primary_color_tl]
          (\l_primary_note_start_int - 21, \ambitusMinY)
          rectangle
          ++(\l_primary_note_end_int - \l_primary_note_start_int + 1, \ambitusBarHeight);

        \node[anchor=north~ west, text=white] (\l_low_node_name_tl) at (\l_primary_note_start_int - 21, \ambitusMinY) {
          \tl_if_empty:NTF \l_primary_start_pitch_class_label_tl {
            \noteName{\l_primary_note_start_int}
          } {
            \l_primary_start_pitch_class_label_tl \noteOctave{\l_primary_note_start_int}
          }
        };

        \bool_if:nT { \l_secondary_note_start_is_below_primary_bool } {
          \node[anchor=north~ west, text=white] (\l_low_node_name_tl) at (\l_secondary_note_start_int - 21, \ambitusMinY) {
            \noteName{\l_secondary_note_start_int}
          };
        }

        \node[anchor=north~ west, text=white] (\l_high_node_name_tl) at (\l_primary_note_end_int - 21, \ambitusMinY) {
          \int_compare:nNnT { \l_primary_note_start_int } < { \l_primary_note_end_int } {
            \tl_if_empty:NTF \l_primary_end_pitch_class_label_tl {
              \noteName{\l_primary_note_end_int}
            } {
              \l_primary_end_pitch_class_label_tl \noteOctave{\l_primary_note_end_int}
            }
          }
        };

        \bool_if:nT { \l_secondary_note_end_is_above_primary_bool } {
          \node[anchor=north~ west, text=white] (\l_high_node_name_tl) at (\l_secondary_note_end_int - 21, \ambitusMinY) {
            \noteName{\l_secondary_note_end_int}
          };
        }

        \tl_if_empty:NF \l_long_name_tl {
          \node[base~ left=of~ \l_low_node_name_tl, xshift=\labelOffset, instrument~ label] { \l_long_name_tl };
        }
        \tl_if_empty:NF \l_short_name_tl {
          \node[base~ right=of~ \l_high_node_name_tl, xshift=-\labelOffset] { \l_short_name_tl };
        }
      \end{scope}
    \end{scope}
  }

  \NewDocumentCommand { \ambitus } { o o m } {
    \ambitusComponent[#1][#2]{#3}

    \pgfmathsetmacro{\ambitusBarOffset}{\ambitusBarOffset + 1 + \ambitusBarSeparation}
    \xdef\ambitusBarOffset{\ambitusBarOffset}
  }

  \ExplSyntaxOff

  \NewDocumentEnvironment{instrumentFamily}{}{
    \def\defaultInstrumentFamilySeparation{1.5}
    \gdef\instrumentFamilySeparation{\defaultInstrumentFamilySeparation}

    \pgfmathsetmacro{\ambitusBarOffset}{\ambitusBarOffset + \instrumentFamilySeparation}
    \xdef\ambitusBarOffset{\ambitusBarOffset}
  }{
    \gdef\instrumentFamilySeparation{\defaultInstrumentFamilySeparation}
  }

  \NewDocumentEnvironment{bracketedAmbitus}{m}{
    \def\ambitusBarSeparation{0.25}

    \newtoggle{didRun}

    \NewCommandCopy\ambitusComponentOriginal\ambitusComponent
    \RenewDocumentCommand\ambitusComponent{oom}{
      \iftoggle{didRun}{
        \ambitusComponentOriginal[##1][##2]{##3, low-node-name=#1-low}
      }{
        \ambitusComponentOriginal[##1][##2]{##3, high-node-name=#1-high}
        \toggletrue{didRun}
      }
    }

    \begin{scope}
  }{
    \end{scope}

    \path[draw=ambitus-bracket-color, overlay]
      let \p1=(#1-low), \p2=(#1-high) in
      (\x1 - 0.25 * \blackKeyWidth, \y1 - \ambitusBarHeight cm) --
      (\x1 - 0.75 * \blackKeyWidth, \y1 - \ambitusBarHeight cm) --
      (\x1 - 0.75 * \blackKeyWidth, \y2 + 0.5 * \ambitusBarHeight cm)
      node[midway, anchor=mid east, xshift=\labelOffset] (#1-midpoint) { #1 } --
      (\x1 - 0.25 * \blackKeyWidth, \y2 + 0.5 * \ambitusBarHeight cm)
      % let \p3=(#1-midpoint) in
      % (\x1 - 0.75 * \blackKeyWidth, \y3) -- (\x1 - \blackKeyWidth, \y3)
      ;

    \RenewCommandCopy\ambitusComponent\ambitusComponentOriginal

    \pgfmathsetmacro{\ambitusBarOffset}{\ambitusBarOffset + 1}
    \xdef\ambitusBarOffset{\ambitusBarOffset}

    \gdef\instrumentFamilySeparation{0.5}
  }

  \NewDocumentEnvironment{compactAmbitus}{}{\def\ambitusBarSeparation{0}}{}

  % Label notes.
  \begin{scope}[anchor=north west]
    \labelCsAndConcertPitch
  \end{scope}

  % Add the top staff.
  \path[overlay]
    let \p1=(midi-note-60) in
    (\noteImageXCoordinate, \y1 - \ambitusBarHeight cm)
    node[anchor=north west] (notes-top) {
      \includegraphics[scale=\noteImageScaleFactor]{notes-top.pdf}
    };

  \node[below=0 of notes-top] {};
  \pgfgetlastxy{\lastx}{\lasty}
  \gdef\ambitusBarOffset{-3.5 - \ambitusBarSeparation - \lasty * 2.54/72.27 / \ambitusBarHeight}

  \begin{scope}[
    minimum height=\ambitusBarHeight cm,
    minimum width=\blackKeyWidth
  ]
    \begin{instrumentFamily}
      \ambitus  [Piccolo]             [Picc.]   {notes={74, 108}}
      \ambitus  [Flute]               [Fl.]     {notes={60,  98}, secondary-notes=59}
      \ambitus  [Alto Flute]          [A.Fl.]   {notes={55,  88}}
      \ambitus  [Oboe]                [Ob.]     {notes={58,  91}, secondary-notes=96}
      \ambitus  [English Horn]        [EH]      {notes={52,  83}}
      \ambitus  [Heckelphone]         [Heck.]   {notes={45,  79}, secondary-notes=84}
      \ambitus  [E♭ Clarinet]         [E♭ Cl.]  {notes={55,  95}}
      \ambitus  [B♭ Clarinet]         [B♭ Cl.]  {notes={50,  91}}
      \ambitus  [E♭ Alto Clarinet]    [A.Cl.]   {notes={43,  79}, secondary-notes={42, 84}}
      \ambitus  [Bass Clarinet]       [B.Cl.]   {notes={34,  79}, secondary-notes=83}
      \begin{scope}[instrument label/.append style={anchor=north west, xshift=-\labelOffset, yshift=-0.2*\ambitusBarHeight cm}]
        \ambitus[Contrabass Clarinet] [Cb. Cl.] {notes={22,  67}}
      \end{scope}
      \ambitus  [Bassoon]             [Bsn.]    {notes={34,  75}, secondary-notes=77}
      \begin{scope}[instrument label/.append style={anchor=north west, xshift=-\labelOffset, yshift=-0.2*\ambitusBarHeight cm}]
        \ambitus[Contrabassoon]       [Cbsn.]   {notes={22,  60}, secondary-notes=21}
      \end{scope}
    \end{instrumentFamily}

    \begin{instrumentFamily}
      \ambitus[Soprano Saxophone]  [S. Sax.]   {notes={56, 94}}
      \ambitus[Alto Saxophone]     [A. Sax.]   {notes={49, 87}}
      \ambitus[Tenor Saxophone]    [T. Sax.]   {notes={44, 82}}
      \ambitus[Baritone Saxophone] [Bar. Sax.] {notes={37, 75}, secondary-notes=36}

      \path
        let \p1=(Soprano Saxophone-high) in
        (\x1 + 3.75 * \blackKeyWidth, \y1 + \ambitusBarHeight cm)
        node[annotation, anchor=mid west, text width=12.75*\blackKeyWidth] {
          Higher notes are possible on wind instruments, but increasingly difficult to sound and not always available in sample sets.
        };
    \end{instrumentFamily}

    \begin{instrumentFamily}
      \ambitus[French Horn]        [Hn.]           {notes={31, 77}}
      \ambitus[B♭ Piccolo Trumpet] [B♭ Picc. Tpt.] {notes={64, 91}, secondary-notes=62}
      \ambitus[C Trumpet]          [C Tpt.]        {notes={54, 84}, secondary-notes=53}
      \ambitus[B♭ Trumpet]         [B♭ Tpt.]       {notes={52, 82}, secondary-notes=51}
      \ambitus[B♭ Bass Trumpet]    [B♭ B. Tpt.]    {notes={40, 72}}
      \ambitus[Trombone]           [Tbn.]          {notes={40, 77}}
      \ambitus[Bass Trombone]      [B. Tbn.]       {notes={34, 77}}
      \ambitus[B♭ Wagner Tuba]     [B♭ Wag.]       {notes={40, 77}}
      \ambitus[F Wagner Tuba]      [F Wag.]        {notes={35, 77}}
      \ambitus[B♭ Baritone]        [B♭ Bar.]       {notes={40, 72}}
      \ambitus[B♭ Euphonium]       [B♭ Euph.]      {notes={35, 72}}
      \ambitus[C Tuba]             [C Tuba]        {notes={26, 64}, secondary-notes=24}
      \ambitus[B♭ Tuba]            [B♭ Tuba]       {notes={22, 63}}

      \path
        let \p1=(C Trumpet-low) in
        (0, \y1)
        node[annotation, anchor=north west, text width=7.5*\blackKeyWidth] {
          Lower pedal tones are possible on brass instruments.
        };
    \end{instrumentFamily}
  \end{scope}

  % Draw the keyboard.
  \pgfmathsetmacro{\blackKeyHeight}{0.625 * \keyboardHeight}
  \pgfmathsetmacro{\whiteToBlackKeyDistance}{\keyboardHeight - \blackKeyHeight}
  \pgfmathsetmacro{\lastOctave}{div(\totalNoteCount, 12) - 1}
  \node[below=0 of B-flat Tuba-low] {};
  \pgfgetlastxy{\lastx}{\lasty}
  \begin{scope}[yshift=\lasty - \ambitusBarHeight cm]
    \begin{scope}[draw=black-key-color]
      \draw[fill=white] (0, 0) rectangle (\totalNoteCount, \keyboardHeight);

      \NewDocumentCommand\AToCSharpKeyPaths{}{
        ++( 1.5, \keyboardHeight) --        ++(0, -\whiteToBlackKeyDistance)
        ++(-0.5, 0)               rectangle ++(1, -\blackKeyHeight) % A♯/B♭
        ++( 1  , \keyboardHeight) --        ++(0, -\keyboardHeight)
        ++( 1.5, \keyboardHeight) --        ++(0, -\whiteToBlackKeyDistance)
        ++(-0.5, 0)               rectangle ++(1, -\blackKeyHeight) % C♯/D♭
      }
      \filldraw[fill=black-key-color] (0, 0)
      \foreach \octave in { 0, ..., \lastOctave } {
        \AToCSharpKeyPaths
        ++( 1.5, \keyboardHeight) --        ++(0, -\whiteToBlackKeyDistance)
        ++(-0.5, 0)               rectangle ++(1, -\blackKeyHeight) % D♯/E♭
        ++( 1  , \keyboardHeight) --        ++(0, -\keyboardHeight)
        ++( 1.5, \keyboardHeight) --        ++(0, -\whiteToBlackKeyDistance)
        ++(-0.5, 0)               rectangle ++(1, -\blackKeyHeight) % F♯/G♭
        ++( 1.5, \keyboardHeight) --        ++(0, -\whiteToBlackKeyDistance)
        ++(-0.5, 0)               rectangle ++(1, -\blackKeyHeight) % G♯/A♭
      }
      \AToCSharpKeyPaths;

      % Label notes.
      \begin{scope}[anchor=north west, yshift=-0.3ex]
        \labelCsAndConcertPitch
      \end{scope}
    \end{scope}

    % Add MIDI note numbers.
    \begin{scope}[
      anchor=mid west,
      minimum width=\blackKeyWidth,
    ]
      \foreach \noteNumber in { 21, ..., 110 } {
        \pgfmathsetmacro{\pitchNumber}{mod(\noteNumber, 12)}
        \pgfmathparse{\pitchNumber ==  1 || \pitchNumber ==  3 || \pitchNumber ==  6 || \pitchNumber ==  8 || \pitchNumber == 10}
        \ifnum\pgfmathresult>0\relax
          \def\noteNumberColor{black-note-number-color}
        \else
          \def\noteNumberColor{white-note-number-color}
        \fi
        \node[text=\noteNumberColor] at (\noteNumber - 21, 0.5 * \keyboardHeight) { \noteNumber };
      }
    \end{scope}

    \node[
      align=center,
      anchor=north east,
      inner xsep=1pt,
      minimum height=\keyboardHeight cm,
      overlay,
      text=white-note-number-color
    ] (MIDI note label) at (0, 0) { MIDI\\[-0.5ex] note nos. };
  \end{scope}

  \node[below=0 of MIDI note label, overlay] {};
  \pgfgetlastxy{\lastx}{\lasty}
  \gdef\ambitusBarOffset{-\ambitusBarSeparation - \lasty * 2.54/72.27 / \ambitusBarHeight}

  \begin{scope}[
    minimum height=\ambitusBarHeight cm,
    minimum width=\blackKeyWidth
  ]
    \begin{instrumentFamily}
      \begin{bracketedAmbitus}{Timpani}
        \begin{compactAmbitus}
          \ambitus{notes={53, 61}, color=secondary-ambitus-color}
          \ambitus{notes={50, 57}}
          \ambitus{notes={46, 53}}
          \ambitus{notes={41, 48}}
          \ambitus{notes={36, 45}, color=secondary-ambitus-color}
        \end{compactAmbitus}
      \end{bracketedAmbitus}
      \pgfmathsetmacro{\ambitusBarOffset}{\ambitusBarOffset - 4 - 2 * \ambitusBarSeparation}
      \xdef\ambitusBarOffset{\ambitusBarOffset}
      \ambitus[Crotales]      [Crot.]    {notes={84, 108}}
      \ambitus[Glockenspiel]  [Glck.]    {notes={79, 108}, secondary-notes=72}
      \ambitus[Celesta]       [Cel.]     {notes={60, 108}}
      \ambitus[Tubular Bells] [T. Bells] {notes={60,  77}, secondary-notes={52, 79}}
      \ambitus[Vibraphone]    [Vib.]     {notes={53,  89}, secondary-notes=48}
      \ambitus[Xylophone]     [Xyl.]     {notes={65, 108}, secondary-notes=58}
      \ambitus[Marimba]       [Mar.]     {notes={48,  96}, secondary-notes=36}
      \ambitus[Harp]          [Harp]     {notes={23, 104}, note-labels={C♭, G♯}}
      \ambitus[Piano]         [Pno.]     {notes={21, 108}}
    \end{instrumentFamily}

    \begin{instrumentFamily}
      \begin{bracketedAmbitus}{Violin}
        \ambitusComponent[Vln. harmonics]{notes=67} \ambitusComponent{notes=74} \ambitus{notes={79, 108}, secondary-notes=110}
        \begin{compactAmbitus}
          \ambitus{notes={76, 100}}
          \ambitus{notes={69,  93}}
          \ambitus{notes={62,  86}}
          \ambitus{notes={55,  79}}
        \end{compactAmbitus}
      \end{bracketedAmbitus}

      \begin{bracketedAmbitus}{Viola}
        \ambitusComponent[Vla. harmonics]{notes=60} \ambitusComponent{notes=67} \ambitus{notes={72, 100}, secondary-notes=101}
        \begin{compactAmbitus}
          \ambitus{notes={69, 93}}
          \ambitus{notes={62, 86}}
          \ambitus{notes={55, 79}}
          \ambitus{notes={48, 72}}
        \end{compactAmbitus}
      \end{bracketedAmbitus}

      \begin{bracketedAmbitus}{Cello}
        \ambitusComponent[Vcl. harmonics]{notes=48} \ambitusComponent{notes=55} \ambitus{notes={60, 89}, secondary-notes=95}
        \begin{compactAmbitus}
          \ambitus{notes={57, 81}, secondary-notes=82}
          \ambitus{notes={50, 67}, secondary-notes=74}
          \ambitus{notes={43, 60}, secondary-notes=67}
          \ambitus{notes={36, 53}, secondary-notes=60}
        \end{compactAmbitus}

        \path
          let \p1=(Cello-high) in
          (96.5 - 21, \y1 + 0.5*\ambitusBarHeight cm)
          node[annotation, anchor=mid west, text width=14.5*\blackKeyWidth] {
            Higher harmonics are possible on stringed instruments, but not always available in sample sets.
          };
      \end{bracketedAmbitus}

      \begin{bracketedAmbitus}{Contrabass}
        \ambitusComponent[Cb. harmonics]{notes=40} \ambitusComponent{notes=45} \ambitusComponent{notes=47} \ambitusComponent{notes=50} \ambitus{notes={52, 84}, secondary-notes=89}
        \begin{compactAmbitus}
          \ambitus{notes={43, 55}, secondary-notes=67}
          \ambitus{notes={38, 50}, secondary-notes=62}
          \ambitus{notes={33, 45}, secondary-notes=57}
          \ambitus{notes={28, 40}, secondary-notes={24, 52}}
        \end{compactAmbitus}
      \end{bracketedAmbitus}
    \end{instrumentFamily}
  \end{scope}

  % Add the bottom staff.
  \path[overlay]
    let \p1=(Contrabass-low) in
    (\noteImageXCoordinate, \y1)
    node[anchor=north west] (notes-bottom) {
      \includegraphics[scale=\noteImageScaleFactor]{notes-bottom.pdf}
    };

  \node[below=0 of notes-bottom] {};
  \pgfgetlastxy{\lastx}{\lasty}
  \pgfmathsetmacro{\frequencyY}{-\lasty * 2.54/72.27}

  \begin{scope}[anchor=south west, yshift=-\frequencyY cm + 0.1ex]
    \labelCsAndConcertPitch[0]
  \end{scope}

  % Add frequencies.
  \pgfkeys{/pgf/number format/.cd,
    assume math mode=true, % Don’t switch to math mode (which would use the wrong font).
    fixed,                 % Round numbers to a fixed number of digits after the decimal point.
    precision=1            % Round to the nearest tenth.
  }

  % The highest note is the D five half-steps above the A at 3520 Hz, so in most
  % DAWs using 12-EDO tuning, the fundamental frequency of the D will be
  % 2^(5/12) * 3520 Hz. Use a LaTeX3 floating point expression because the PGF
  % math engine is a bit too imprecise to compute this.
  \ExplSyntaxOn
  \def\maxFrequency{\fp_to_decimal:n {2^(5/12) * 3520}}
  \ExplSyntaxOff

  \def\minFrequency{27.5}

  \newtoggle{didPrintFrequencyUnit}
  \foreach \frequency in {
    % Frequencies of A notes
    \minFrequency, 55, 110, 220, 440, 880, 1760, 3520,
    % ISO 266 frequencies
      31.5, 40,  50,  63,  80,  100,  125,  160,  200,  250,
     315,  400, 500, 630, 800, 1000, 1250, 1600, 2000, 2500,
    3150, 4000,
    % Max frequency
    \maxFrequency
  } {
    \pgfmathparse{
      89 * ln(\frequency / \minFrequency) / ln(\maxFrequency / \minFrequency) + 0.5
      % This is a simplification of
      %   (ln(\frequency) - ln(\minFrequency)) * (89.5 - 0.5) / (ln(\maxFrequency) - ln(\minFrequency)) + 0.5
    }
    \draw[frequency-label-color, overlay] (\pgfmathresult, \frequencyY) -- ++(0, 0.05) node[anchor=north, yshift=-0.2ex] {
      \pgfmathprintnumber{\frequency} \iftoggle{didPrintFrequencyUnit}{}{Hz\global\toggletrue{didPrintFrequencyUnit}}
    };
  }

  \begin{scope}[on background layer]
    % Draw white note guides.
    \NewDocumentCommand\AToDGuidePaths{}{
      rectangle ++(1, -\frequencyY) ++(1, \frequencyY) % A
      rectangle ++(1, -\frequencyY) ++(2, \frequencyY) % B
      rectangle ++(1, -\frequencyY) ++(1, \frequencyY) % D
    }
    \def\EFSeparation{0.05}
    \fill[fill=white-note-guide-color]
      (0, \frequencyY)
      \foreach \octave in { 0, ..., \lastOctave } {
        \AToDGuidePaths
        rectangle ++(1 - \EFSeparation, -\frequencyY) ++(2 * \EFSeparation, \frequencyY) % E
        rectangle ++(1 - \EFSeparation, -\frequencyY) ++(1,                 \frequencyY) % F
        rectangle ++(1,                 -\frequencyY) ++(1,                 \frequencyY) % G
      }
      \AToDGuidePaths;

    % Draw black note guides.
    \NewDocumentCommand\BFlatToCSharpGuidePaths{}{
      rectangle ++(1, -\frequencyY) ++(2, \frequencyY) % A♯/B♭
      rectangle ++(1, -\frequencyY) ++(1, \frequencyY) % C♯/D♭
    }
    \fill[fill=black-note-guide-color]
      (1, \frequencyY)
      \foreach \octave in { 0, ..., \lastOctave } {
        \BFlatToCSharpGuidePaths
        rectangle ++(1, -\frequencyY) ++(2, \frequencyY) % D♯/E♭
        rectangle ++(1, -\frequencyY) ++(1, \frequencyY) % F♯/G♭
        rectangle ++(1, -\frequencyY) ++(1, \frequencyY) % G♯/A♭
      }
      \BFlatToCSharpGuidePaths;
  \end{scope}
\end{tikzpicture}

\clearpage
\pagestyle{plain}

\begin{multicols*}{4}
  \scriptsize
  \addfontfeature{Numbers=OldStyle}

  \section*{Instrument Transpositions}

  \begin{tabular}{@{}lll@{}} \toprule
    Instrument          & To Concert Pitch        \\ \midrule
    Piccolo             & octave up               \\
    Alto Flute          & 4th down                \\
    English Horn        & 5th down                \\
    Heckelphone         & octave down             \\
    E♭ Clarinet         & minor~3rd up            \\
    B♭ Clarinet         & 2nd down                \\
    E♭ Alto Clarinet    & minor 6th down          \\
    Bass Clarinet       & 2nd + octave down       \\
    Contrabass Clarinet & 2nd + 2 octaves down    \\
    Contrabassoon       & octave down             \\ \midrule
    Soprano Saxophone   & 2nd down                \\
    Alto Saxophone      & minor 6th down          \\
    Tenor Saxophone     & 2nd + octave down       \\
    Baritone Saxophone  & minor 6th + octave down \\ \midrule
    French Horn in F    & \begin{tabular}{@{}l@{}}
                            5th down, treble \& new bass clef \\
                            4th up, old bass clef
                          \end{tabular}           \\
    B♭ Piccolo Trumpet  & minor 7th up            \\
    B♭ Trumpet          & 2nd down                \\
    B♭ Bass Trumpet     & 2nd + octave down       \\
    B♭ Wagner Tuba      & 2nd down                \\
    F Wagner Tuba       & 5th down                \\
    B♭ Baritone         & 2nd + octave down       \\
    B♭ Euphonium        & 2nd + octave down       \\ \midrule
    Crotales            & 2 octaves up            \\
    Glockenspiel        & 2 octaves up            \\
    Celesta             & octave up               \\
    Xylophone           & octave up               \\ \midrule
    Contrabass          & octave down             \\ \bottomrule
  \end{tabular}


  \section*{Note Names}

  \def\rolandOctaves{“Roland” octaves (used in Ableton Live and Native Instruments Kontakt, concert pitch is {\addfontfeature{Numbers=Lining}A3})}
  \def\yamahaOctaves{“Yamaha” octaves (Apple Logic’s default and used in FabFilter {\addfontfeature{Numbers=Lining}Pro-Q~3}, concert pitch is {\addfontfeature{Numbers=Lining}A4})}

  Note names have
  \ifnum\octaveOffset>1\relax
    \rolandOctaves. Add 1 for \yamahaOctaves.
  \else
    \yamahaOctaves. Subtract 1 for \rolandOctaves.
  \fi


  \section*{Page Numbers of References}

  For more information about instruments and orchestration:

  \newcolumntype{R}{>{\addfontfeature{Numbers={Lining,Monospaced}}}r}

  \def\thinrule{\specialrule{\cmidrulewidth}{0.5\aboverulesep}{0.5\belowrulesep}}

  \newlength\instrumentWidth
  \settowidth{\instrumentWidth}{Contrabass Clarinet}

  \addtolength{\tabcolsep}{-0.4em}

  % A longtable environment would be preferred, but longtable can’t be used in a
  % multicols environment.
  \begin{tabular}[t]{@{}p{\instrumentWidth}RRRR@{}} \toprule
                        & \multicolumn{4}{c}{Page Numbers} \\ \cmidrule(){2-5}
    Instrument          & \citeauthor{adler}    & \citeauthor{blatter}                     & \citeauthor{forsyth}    & \citeauthor{sevsay}    \\ \midrule
    Piccolo             & \notecite[198]{adler} & \multirow{3}{*}{\notecite[90]{blatter}}  & \notecite[198]{forsyth} & \notecite[76]{sevsay}  \\
    Flute               & \notecite[189]{adler} &                                          & \notecite[182]{forsyth} & \notecite[75]{sevsay}  \\
    Alto Flute          & \notecite[201]{adler} &                                          & \notecite[196]{forsyth} & \notecite[77]{sevsay}  \\ \thinrule
    Oboe                & \notecite[204]{adler} & \multirow{3}{*}{\notecite[98]{blatter}}  & \notecite[204]{forsyth} & \notecite[78]{sevsay}  \\
    English Horn        & \notecite[209]{adler} &                                          & \notecite[220]{forsyth} & \notecite[79]{sevsay}  \\
    Heckelphone         & \notecite[216]{adler} &                                          & \notecite[228]{forsyth} & \notecite[80]{sevsay}  \\ \thinrule
    E♭ Clarinet         & \notecite[224]{adler} & \multirow{5}{*}{\notecite[105]{blatter}} & \notecite[278]{forsyth} & \notecite[82]{sevsay}  \\
    B♭ Clarinet         & \notecite[217]{adler} &                                          & \notecite[251]{forsyth} & \notecite[80]{sevsay}  \\
    E♭ Alto Clarinet    & \notecite[228]{adler} &                                          & \notecite[282]{forsyth} & \notecite[83]{sevsay}  \\
    Bass Clarinet       & \notecite[225]{adler} &                                          & \notecite[272]{forsyth} & \notecite[82]{sevsay}  \\
    Contrabass Clarinet & \notecite[230]{adler} &                                          & \notecite[286]{forsyth} & \notecite[84]{sevsay}  \\ \thinrule
    Bassoon             & \notecite[235]{adler} & \multirow{2}{*}{\notecite[116]{blatter}} & \notecite[229]{forsyth} & \notecite[84]{sevsay}  \\
    Contrabassoon       & \notecite[240]{adler} &                                          & \notecite[246]{forsyth} & \notecite[85]{sevsay}  \\ \midrule
    Saxophones          & \notecite[231]{adler} & \notecite[126]{blatter}                  & \notecite[166]{forsyth} & \notecite[87]{sevsay}  \\ \bottomrule
  \end{tabular}

  % \vfill\null
  \columnbreak

  \emph{(Page numbers of references, cont.)}\\
  \begin{tabular}[t]{@{}p{\instrumentWidth}RRRR@{}} \toprule
                        & \multicolumn{4}{c}{Page Numbers} \\ \cmidrule(){2-5}
    Instrument          & \citeauthor{adler}    & \citeauthor{blatter}                     & \citeauthor{forsyth}    & \citeauthor{sevsay}    \\ \midrule
    French Horn         & \notecite[337]{adler} & \notecite[148]{blatter}                  & \notecite[109]{forsyth} & \notecite[96]{sevsay}  \\
    Trumpets            & \notecite[357]{adler} & \notecite[159]{blatter}                  & \notecite[89]{forsyth}  & \notecite[101]{sevsay} \\
    Trombones           & \notecite[368]{adler} & \notecite[169]{blatter}                  & \notecite[133]{forsyth} & \notecite[103]{sevsay} \\
    Wagner Tubas        & —                     & \notecite[148]{blatter}                  & \notecite[153]{forsyth} & \notecite[111]{sevsay} \\
    Tubas               & \notecite[377]{adler} & \notecite[178]{blatter}                  & \notecite[151]{forsyth} & \notecite[108]{sevsay} \\ \midrule
    Timpani             & \notecite[485]{adler} & \notecite[209]{blatter}                  & \notecite[41]{forsyth}  & \notecite[167]{sevsay} \\ \midrule
    Crotales            & \notecite[481]{adler} & \notecite[206]{blatter}                  & —                       & \notecite[161]{sevsay} \\
    Glockenspiel        & \notecite[479]{adler} & \notecite[205]{blatter}                  & \notecite[60]{forsyth}  & \notecite[150]{sevsay} \\
    Tubular Bells       & \notecite[480]{adler} & \notecite[205]{blatter}                  & \notecite[53]{forsyth}  & \notecite[155]{sevsay} \\
    Celesta             & \notecite[528]{adler} & \notecite[206]{blatter}                  & \notecite[64]{forsyth}  & \notecite[245]{sevsay} \\
    Vibraphone          & \notecite[477]{adler} & \notecite[205]{blatter}                  & —                       & \notecite[152]{sevsay} \\
    Xylophone           & \notecite[475]{adler} & \notecite[204]{blatter}                  & \notecite[66]{forsyth}  & \notecite[147]{sevsay} \\
    Marimba             & \notecite[476]{adler} & \notecite[204]{blatter}                  & —                       & \notecite[149]{sevsay} \\ \midrule
    Harp                & \notecite[95]{adler}  & \notecite[252]{blatter}                  & \notecite[461]{forsyth} & \notecite[194]{sevsay} \\
    Piano               & \notecite[521]{adler} & \notecite[242]{blatter}                  & —                       & \notecite[230]{sevsay} \\ \midrule
    Violin              & \notecite[57]{adler}  & \notecite[49, 441]{blatter}              & \notecite[303]{forsyth} & \multirow{4}{*}{\notecite[3]{sevsay}} \\
    Viola               & \notecite[71]{adler}  & \notecite[56, 442]{blatter}              & \notecite[381]{forsyth} &                        \\
    Cello               & \notecite[81]{adler}  & \notecite[60, 443]{blatter}              & \notecite[409]{forsyth} &                        \\
    Contrabass          & \notecite[89]{adler}  & \notecite[67, 444]{blatter}              & \notecite[436]{forsyth} &                        \\ \bottomrule
  \end{tabular}

  \nocite{vienna-academy}

  \renewcommand*\bibfont{\scriptsize}
  \setlength\bibhang{10pt}
  \setlength\bibitemsep{\parskip}
  \printbibliography


  \section*{Legal}

  % https://www.ableton.com/en/legal/trademark-list/
  Ableton and Live are trademarks of Ableton AG.

  % https://www.apple.com/legal/intellectual-property/trademark/appletmlist.html
  Apple and Logic are trademarks of Apple Inc., registered in the U.S. and other countries and regions.

  % https://tsdr.uspto.gov/#caseNumber=79394860&caseSearchType=US_APPLICATION&caseType=DEFAULT&searchType=statusSearch
  FabFilter is a trademark of FabFilter BV.

  % https://tsdr.uspto.gov/#caseNumber=73083201&caseSearchType=US_APPLICATION&caseType=DEFAULT&searchType=statusSearch
  Roland is a registered trademark of Roland Corporation.

  % https://tsdr.uspto.gov/#caseNumber=77642839&caseSearchType=US_APPLICATION&caseType=DEFAULT&searchType=statusSearch
  Vienna Symphonic Library is a registered trademark of Vienna Symphonic Library GmbH.

  % https://www.yamaha.com/paragon/trademarkguidelines/
  Yamaha is a registered trademark of Yamaha Corporation.
\end{multicols*}

\end{document}
